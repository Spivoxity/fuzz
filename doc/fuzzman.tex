%
% fuzzman.tex
%
% This file is part of fuzz2000
% Copyright (c) 1982--2006 J. M. Spivey
% All rights reserved
%
% Redistribution and use in source and binary forms, with or without
% modification, are permitted provided that the following conditions are met:
%
% 1. Redistributions of source code must retain the above copyright notice,
%    this list of conditions and the following disclaimer.
% 2. Redistributions in binary form must reproduce the above copyright notice,
%    this list of conditions and the following disclaimer in the documentation
%    and/or other materials provided with the distribution.
% 3. The name of the author may not be used to endorse or promote products
%    derived from this software without specific prior written permission.
%
% THIS SOFTWARE IS PROVIDED BY THE AUTHOR ``AS IS'' AND ANY EXPRESS OR
% IMPLIED WARRANTIES, INCLUDING, BUT NOT LIMITED TO, THE IMPLIED WARRANTIES
% OF MERCHANTABILITY AND FITNESS FOR A PARTICULAR PURPOSE ARE DISCLAIMED.
% IN NO EVENT SHALL THE AUTHOR BE LIABLE FOR ANY DIRECT, INDIRECT, INCIDENTAL,
% SPECIAL, EXEMPLARY, OR CONSEQUENTIAL DAMAGES (INCLUDING, BUT NOT LIMITED TO,
% PROCUREMENT OF SUBSTITUTE GOODS OR SERVICES; LOSS OF USE, DATA, OR PROFITS;
% OR BUSINESS INTERRUPTION) HOWEVER CAUSED AND ON ANY THEORY OF LIABILITY,
% WHETHER IN CONTRACT, STRICT LIABILITY, OR TORT (INCLUDING NEGLIGENCE OR
% OTHERWISE) ARISING IN ANY WAY OUT OF THE USE OF THIS SOFTWARE, EVEN IF
% ADVISED OF THE POSSIBILITY OF SUCH DAMAGE.
% 

\documentstyle[fuzz209,zrmref]{a5man}
\makeindex
%
% hacks.tex
%
% This file is part of fuzz2000
% Copyright (c) 1982--2006 J. M. Spivey
% All rights reserved
%
% Redistribution and use in source and binary forms, with or without
% modification, are permitted provided that the following conditions are met:
%
% 1. Redistributions of source code must retain the above copyright notice,
%    this list of conditions and the following disclaimer.
% 2. Redistributions in binary form must reproduce the above copyright notice,
%    this list of conditions and the following disclaimer in the documentation
%    and/or other materials provided with the distribution.
% 3. The name of the author may not be used to endorse or promote products
%    derived from this software without specific prior written permission.
%
% THIS SOFTWARE IS PROVIDED BY THE AUTHOR ``AS IS'' AND ANY EXPRESS OR
% IMPLIED WARRANTIES, INCLUDING, BUT NOT LIMITED TO, THE IMPLIED WARRANTIES
% OF MERCHANTABILITY AND FITNESS FOR A PARTICULAR PURPOSE ARE DISCLAIMED.
% IN NO EVENT SHALL THE AUTHOR BE LIABLE FOR ANY DIRECT, INDIRECT, INCIDENTAL,
% SPECIAL, EXEMPLARY, OR CONSEQUENTIAL DAMAGES (INCLUDING, BUT NOT LIMITED TO,
% PROCUREMENT OF SUBSTITUTE GOODS OR SERVICES; LOSS OF USE, DATA, OR PROFITS;
% OR BUSINESS INTERRUPTION) HOWEVER CAUSED AND ON ANY THEORY OF LIABILITY,
% WHETHER IN CONTRACT, STRICT LIABILITY, OR TORT (INCLUDING NEGLIGENCE OR
% OTHERWISE) ARISING IN ANY WAY OUT OF THE USE OF THIS SOFTWARE, EVEN IF
% ADVISED OF THE POSSIBILITY OF SUCH DAMAGE.
% 

% This file contains hacks needed for both the fuzz manual and
% the free guide to the fuzz style option

\makeatletter % Oh-ho, prepare for a big dose of lark's vomit fondue

% The verbatim environment is indented like a quote.
% There are also numverb and demo environments.

\begingroup \catcode `|=0 \catcode `[= 1
\catcode`]=2 \catcode `\{=12 \catcode `\}=12
\catcode`\\=12
|gdef|@demo#1\gives[#1|gives]
|gdef|@xnumverb#1\end{numverb}[#1|end[numverb]]
|endgroup

\def\@verbatim{\quote\if@minipage\else\vskip\parskip\fi
\leftskip\@totalleftmargin\rightskip\z@
\parindent\z@\parfillskip\@flushglue\parskip\z@
\@tempswafalse \def\par{\if@tempswa\hbox{}\fi\@tempswatrue\@@par}
\obeylines \tt \catcode``=13 \@noligs \let\do\@makeother \dospecials}

\def\verbatim{\@verbatim \frenchspacing\@vobeyspaces \@xverbatim}
\let\endverbatim=\endquote

\def\numverb{\@verbatim\small\tt\@numlines
  \frenchspacing\@vobeyspaces \@xnumverb}
\let\endnumverb=\endquote

\newcount\@lineno
\newtoks\@hack@ % Save list env's everypar
\def\@numlines{\@lineno=0 \@hack@=\everypar
  \everypar{{\the\@hack@}\@hack@{}%
    \advance\@lineno by1 \llap{{\tiny\rm \the\@lineno}\enspace}}}

\def\demo{\predisplaypenalty=10000\preboxpenalty=10000
	\begingroup\@verbatim \frenchspacing\@vobeyspaces \@demo}
\def\gives{\endverbatim\endgroup\@ifnextchar*{\@sgives}{\@gives}}
\def\@gives{\nobreak}
\def\@sgives*{\nobreak\abovedisplayskip=0pt}
\def\enddemo{\global\@ignoretrue}

% The symtab environment is used for 2-column tables of 
% mathematical symbols.  (cf. TeXbook, Appendix D, p. 397.)

\def\symtab{\par \penalty10000
  \vskip-\prevdepth \vskip\dp\strutbox \vskip\medskipamount
  \hrule \nobreak \vskip-0.4pt \vskip\medskipamount
  \setbox0=\vbox\bgroup \def\\{\cr} \symalign}

\def\symalign{\dimen0=\linewidth \divide\dimen0 by2
  \halign to\dimen0\bgroup 
    \strut\ooalign{$##$\hfil\cr\qquad\hfil\cr}\hfil
    &\quad##\hfil\tabskip=0pt plus1fil\cr}

 % This trick allows different widths in the 2 columns
\def\changewidth{\crcr\egroup\symalign}

\def\endsymtab{\crcr\egroup\egroup
  \dimen0=\ht0 \divide\dimen0 by2 \advance\dimen0 by\ht\strutbox
  \splittopskip=\ht\strutbox \vbadness=10000
  \setbox1=\vsplit0 to\dimen0
  \hbox to\linewidth{\valign{##\cr\unvbox1\cr\unvbox0\cr\noalign{\hfil}}}
  \nobreak \vskip\medskipamount \vskip-0.4pt
  \hrule \penalty0 \vskip\medskipamount \prevdepth=\dp\strutbox
  {\parskip=0pt\noindent}\global\@ignoretrue}

% Cross-references to ZRM are surrounded by \ifmanual ... \fi to
% save distributing the ZRM aux file.
\newif\ifmanual \manualfalse

% A few bits and pieces
\let\Fuzz=\fuzz
\def\FUZZ{{\protect\LARGE\it f\kern0.1em}{\protect\Large\sc uzz}}
\def\ZRM{{\rm ZRM}}
\def\lopt{{\it [\/}}
\def\ropt{{\it ]\/}}
\def\SliTeX{{\sc Sli}\TeX}
\def\emTeX{{\sc em}\TeX}

\makeatother % sanity is restored

\manualtrue

\def\frac#1/#2{\leavevmode\kern.1em
  \raise.5ex\hbox{\the\scriptfont0 #1}\kern-.1em
  /\kern-.15em\lower.25ex\hbox{\the\scriptfont0 #2}}

\font\logo=logo10
\def\METAFONT{\hbox{\logo METAFONT}}

\begin{document}
\markboth{The \protect\fuzz\ manual}{}

%% Folio 1: Title page
\thispagestyle{empty}
\font\mytwfvit=cmti10 scaled\magstep5
\vbox to36pt{\vfill
  {\center\huge\sf The {\mytwfvit f\kern0.1em}{\sc uzz} Manual\par}}
\vskip24pt
{\center\Large\sf Mike Spivey\par}
\vskip24pt
{\center\Large\sf Second Edition\par}
\vfill
{\flushright The Spivey Partnership\par}
\bigskip
{\flushright 10 Warneford Road\\
  Oxford\ \ {\small OX4 1LU}\\England\par}
\clearpage
\thispagestyle{empty}
\cleardoublepage

%% Table of contents
\tableofcontents

%% Licence conditions
%% deleted!
\vfill
\noindent Manual and software copyright \copyright\
J. M. Spivey 1988, 1992, 2000
\clearpage
\thispagestyle{empty}
\cleardoublepage

\chapter{Introduction}

\fuzz\ is a collection of tools that help you to format and print Z
specifications and check them for compliance with the Z scope and
type rules. One part of the package is a style option for the
\LaTeX\ type-setting system that defines extra \LaTeX\ commands for
laying out Z specifications, and a font that contains Z's special
symbols.  The other part is a program for analysing and checking
specifications that are written using these commands.

Chapter~\ref{example} of this document is a quick introduction to
\fuzz\ and its features, based on a small example specification.
Chapter~\ref{tex} describes in more detail how to enter your
specification as an input file for \LaTeX\ using the {\tt fuzz} style
option.  Chapters~\ref{tc} to~\ref{mess} describe the \fuzz\
type checker and the messages and reports it can produce.
Chapter~\ref{syntax} is a syntax summary for the Z language as
recognized by the type checker.

In writing this document, I have assumed that you have some basic
knowledge about \LaTeX\ and Z. For information about \LaTeX\ and the
underlying \TeX\ program, you should consult the manuals
\begin{quote}
        Leslie Lamport, {\em \LaTeX{\rm :} A Document Preparation
        System\/},\break
        Addison Wesley, 1985.

        Donald E.~Knuth, {\em The \TeX book}, Addison Wesley, 1984.
\end{quote}
For information about Z, and a description of the scope and type rules
used by the \fuzz\ type checker, you should consult
\begin{quote}
        J. M.~Spivey, {\em The Z Notation: A Reference Manual\/},
        Second edition, Prentice-Hall International, 1992,
\end{quote}
which is referred to in this document by the abbreviation \ZRM.

This manual describes versions of the \fuzz\ package numbered 2.0
or higher; the language supported is that described in the second
edition of the \ZRM.
The \verb/fuzz/ style option is compatible with version
2.99 of \TeX\ and version 2.09 of \LaTeX\ and \SliTeX.

Release 2 of \fuzz\ differs from release 1 in supporting the
additional language constructs added in the second edition of
the~\ZRM, including
\begin{itemize}
\item renaming of schema components.
\item a $\bf let$ construct for local definitions in expressions and
        predicates. 
\item conditional $\bf if\;then\;else$ expressions.
\item piping of operation schemas ($\pipe$).
\end{itemize}
In addition, there is an extended mathematical tool-kit and font of
special symbols, and it is now possible to use \SliTeX\ to make
slides of Z specifications.  Finally, there are improvements in the
type inference algorithm that the \fuzz\ type-checker uses to
calculate the types it displays in error messages and reports.

\chapter{First Steps}\label{example}

The \fuzz\ distribution includes a small example specification in
the file \verb/example.tex/.  Here it is, with the lines
numbered for easy reference:
\begin{numverb}
\documentstyle[fuzz]{article}
\begin{document}
\noindent Let $PERSON$ be the set of all people:
\begin{zed}
        [PERSON].
\end{zed}
A `club' has a set of members and a president, who is 
one of the members:
\begin{schema}{Club}
        members: \power PERSON \\
        president: PERSON
\where
        president \subseteq members
\end{schema}
To enroll somebody in the club, we just add them
to the set of members:
\begin{schema}{Enrol}
        \Delta Club \\
        new?: PERSON
\where
        members' = members \cup new? \\
        president' = president
\end{schema}
The president doesn't change when a new member
is enrolled.
\end{document}
\end{numverb}
\vskip-\lastskip\vfill\eject\noindent %PAGE
This file looks like any other \LaTeX\ manuscript, except that it
includes the style option \verb/fuzz/ (line 1), and it contains three
embedded pieces of Z text, written using the style option's
environments and macros.

The first piece of Z text (lines 4--6) is the declaration of a
basic type $PERSON$. The second and third pieces (lines 9--14 and
17--23) are two schemas: the state space of a very simple membership
register for a club, and an operation that enrolls a new member.

You might like to make a copy of the file and format it with \LaTeX\
by issuing the command:
\begin{quote}
        \verb/latex example/
\end{quote}
Now send the results to a printer. The output should look something
like this:
\begin{quote}
\small
Let $PERSON$ be the set of all people:
\begin{zed}
        [PERSON].
\end{zed}
A `club' has a set of members and a president, who is one of
the members:
\begin{schema}{Club}
        members: \power PERSON \\
        president: PERSON
\where
        president \subseteq members
\end{schema}
To enroll somebody in the club, we just add them to the set of
members:
\begin{schema}{Enrol}
        \Delta Club \\
        new?: PERSON
\where
        members' = members \cup new? \\
        president' = president
\end{schema}
The president doesn't change when a new member is enrolled.
\end{quote}
As you can see, the \verb/zed/ environment produces an ordinary
mathematical display, and the schema environment produces a schema box.

The specification contains a couple of mistakes that can be
found using the \fuzz\ type checker.  So let's run the type checker
on the specification and see what error messages it gives.  You
can run the type checker like this:
\begin{quote}
        \verb/fuzz example.tex/
\end{quote}
The input to the type checker is exactly the same file that you used
as input to \LaTeX.  But instead of printing the document, \fuzz\
scans it, extracts the pieces of text that make up the
formal specification, and checks that they conform to the rules of
the Z language.  Here are the error messages it produces:
\begin{verbatim}
"example.tex", line 13: Type mismatch in left argument
        of infix relation
> Predicate: president \subseteq members
> Arg type:  PERSON
> Expected:  P ?

"example.tex", line 21: Right argument of operator \cup 
        has wrong type
> Expression: members \cup new?
> Arg type:   PERSON
> Expected:   P PERSON
\end{verbatim}
The first error is in the predicate
\[ president \subseteq members. \]
It is supposed to say that the president of the club must be a
member, but the subset relation $\subseteq$ has been used instead of
the membership sign~$\in$.  This is wrong because $president$ is not
a set but a simple element of $PERSON$.  The type checker says that
the left argument of $\subseteq$ must be a set (with type $\power
\hbox{?}$ for some type $\hbox{?}$), but it has found an expression
of type $PERSON$ instead.
The correct thing would be to use the membership sign $\in$ instead
of $\subseteq$, so use your favourite editor to change line 13 of
the file so that it reads
\begin{quote}
        \verb/president \in members/
\end{quote}
The second error is in the expression
\[ members \cup new?. \]
This is supposed to be the set $members$
extended with an additional member $new?$, but it is wrong because
$new?$ is just a $PERSON$, and the union operator $\cup$ expects a
{\em set\/} of elements of $PERSON$.  The type checker knows from
the standard library that $\cup$ expects sets as its arguments, and
it has determined by looking at the left argument $members$ that
this use of $\cup$ is for sets drawn from $PERSON$, but it finds as
the right argument a simple element of $PERSON$.  You should change
line 21 so that it reads
\begin{quote}
        \verb/members' = members \cup \{new?\} \\/
\end{quote}
In the corrected expression, $new?$ is made into a singleton set
before taking the union with $members$, like this:
$members \cup \{new?\}$.
After you have made these changes, the type checker should report no
errors when it checks the document.

In addition to checking for errors, the type checker can produce a
report listing the names defined in your specification and their
types.  You can get this list by giving the command
\begin{quote}
        \verb/fuzz -t example.tex/
\end{quote}
Here is the output:
\begin{verbatim}
Given PERSON

Schema Club
    members: P PERSON
    president: PERSON
End

Schema \Delta Club
    members: P PERSON
    president: PERSON
    members': P PERSON
    president': PERSON
End
\end{verbatim}\vskip-\lastskip\break %PAGE
\begin{verbatim}
Schema Enrol
    members: P PERSON
    president: PERSON
    members': P PERSON
    president': PERSON
    new?: PERSON
End
\end{verbatim}
This output shows that $PERSON$ has been declared as a
basic type, and that three schemas $Club$, $\Delta Club$ and
$Enrol$ have been defined. The definition of $\Delta Club$ was
inserted by the type checker because $\Delta Club$ is used in
$Enrol$ but was given no explicit definition in the specification.
For each schema, the signature is shown after all the inclusions
have been expanded, so you can see that $Enrol$ has five variables
in all.

There is another, much longer, example specification included in the
distribution under the name \verb/tut.tex/; it is essentially the
same as the first chapter of the \ZRM. This file contains examples
of many of the Z constructs, and you may like to look at it as you
read the rest of this manual. The file also shows how the
\fuzz\ style option can be used to print small pieces of program
text.

\chapter{Printing Specifications}\label{tex}

Every \LaTeX\ document begins with a \verb/\documentstyle/
command. If the document contains a Z specification, this command
should include the style option \verb/fuzz/, like
this:\index{|fuzz| style option}
\begin{quote}
        \verb/\documentstyle[12pt,fuzz]{article}/
\end{quote}
The \verb/fuzz/ style option can be used with any of the standard
\LaTeX\ styles, and it can appear either before or after the type-size
option if one is used. It can be combined with most of the standard
style options, but it should not be combined with \verb/fleqn/,
because it already makes provision for setting mathematics flush
left. The \verb/fuzz/ style option can be used with
\SliTeX\index{\SliTeX} to make slides, but only in black and white
at present.

The \verb/fuzz/ style option provides a number of \LaTeX\
environments for making Z boxes of various kinds, and for other
unboxed elements of Z like basic type definitions.  It also
provides many commands that print the fancy symbols of Z using the
special font supplied in the package.  Finally, it provides some
environments that print things, like equational proofs, that are not
properly part of Z but often appear in specification documents.

Section~\ref{boxes} describes the environments that are used for
printing the various boxes and unboxed elements used in Z
specifications; these environments are the ones recognized and
analysed by the \fuzz\ type checker.  Section~\ref{symtabs}
explains how to obtain the symbols that go inside the boxes, and
Section~\ref{finepts} gives some hints on getting the best-looking
layout from \LaTeX\ and the \verb/fuzz/ style option.
Section~\ref{bitsnbobs} explains the environments for printing
miscellaneous displayed formulas.  The chapter closes with a
list (Section~\ref{params}) of the parameters you may adjust to
customize the layout of your specification.

If you want it, the fancy logo `\fuzz' is made by the command
\verb/\fuzz/, which works like the \verb/\LaTeX/ command for
printing \LaTeX's own logo.

\section{Making boxes}\label{boxes}

The Z reference manual lists 
\ifmanual 
  in Sections~\ref{s:spec} and~\ref{s:gendefs} the 
\fi
various kinds of `paragraph' that may
appear as one of the top-level units of a Z specification.  The
\verb/fuzz/ style option provides a \LaTeX\ environment for each
kind of paragraph that comes in a box, and another environment called
\verb/zed/ for the unboxed paragraphs.

\paragraph{Schema definitions%
        \ifmanual\ \rm (\ZRM\ Section~\protect\ref{ss:schemadef})\fi}%
	\index{schema definition: vertical}
To print a schema, just use the \verb/schema/ environment. Here is
an example, showing first the input, then the output from
\LaTeX:\index{|schema| environment}
\begin{demo}
\begin{schema}{BirthdayBook}
    known: \power NAME \\
    birthday: NAME \pfun DATE
\where
    known = \dom birthday
\end{schema}
\gives
\begin{schema}{BirthdayBook}
    known: \power NAME \\
    birthday: NAME \pfun DATE
\where
    known = \dom birthday
\end{schema}
\end{demo}
The name of the schema appears as an argument to the environment,
and (as in all kinds of box) the horizontal dividing line between
declarations and predicates is indicated by
\verb/\where/\index{|\where|}.
Successive lines in the declaration and predicate parts are
separated by the command \verb/\\/\index{|\\| (newline)}.  As usual in
\LaTeX, the actual layout of the input is ignored, and it is only
the explicit command \verb/\\/ that marks the end of a line in the
output.  The Z symbols `$\power$', `$\pfun$' and
`$\dom$' have been entered as the commands \verb/\power/,
\verb/\pfun/ and \verb/\dom/: for a complete list of these commands,
see Section~\ref{symtabs}.

Like the \verb/displaymath/ environment of \LaTeX, the \verb/schema/
environment (and the others we shall come to in a moment) can appear
in the middle of a paragraph of text, and ordinarily should have no blank
lines either before or after it. Blank lines before the environment
are ignored, but blank lines afterwards cause the following text to
begin a new paragraph.

For a schema without a predicate part, the command \verb/\where/ is
simply omitted, as in the following example:
\begin{demo}
\begin{schema}{Document}[CHAR]
    left, right: \seq CHAR
\end{schema}
\gives
\begin{schema}{Document}[CHAR]
    left, right: \seq CHAR
\end{schema}
\end{demo}
This example also shows how to print the generic schemas described in
\ifmanual 
  Section~\ref{ss:genschema} of 
\fi 
the \ZRM.  The formal generic
parameters\index{generic parameter} are an optional argument of the
\verb/schema/ environment.
For compatibility with previous versions of %
\fuzz\index{upward compatibility}, you can also say
\verb/\begin{schema}{Document[CHAR]}/,
making the formal parameters part of the \verb/schema/ environment's
first argument.  The printed effect is the same, and the
type checker recognizes both forms.

If a schema or other box contains more than one predicate below the
line, it often looks better to add a small vertical space between them.
This can be done with the command \verb/\also/\index{|\also|}:
\begin{demo}
\begin{schema}{AddPhone}
    \Delta PhoneDB \\
    name?: NAME \\
    number?: PHONE
\where
    name? \notin known
\also
    phone' = phone \oplus \{name? \mapsto number?\}
\end{schema}
\gives
\begin{schema}{AddPhone}
    \Delta PhoneDB \\
    name?: NAME \\
    number?: PHONE
\where
    name? \notin known
\also
    phone' = phone \oplus \{name? \mapsto number?\}
\end{schema}
\end{demo}

\paragraph{Axiomatic descriptions%
        \ifmanual\ \rm (\ZRM\ Section~\protect\ref{ss:axdef})\fi}%
	\index{axiomatic description}
These are printed with the \verb/axdef/ environment%
\index{|\axdef| environment}\footnote{The
environment should really be called \verb/axdesc/ and not
\verb/axdef/, but that's too hard to pronounce.}.  Here is an
example:
\begin{demo}
\begin{axdef}
    square: \nat \fun \nat
\where
    \forall n: \nat @ \\
\t1     square(n) = n * n
\end{axdef}
\gives
\begin{axdef}
    square: \nat \fun \nat
\where
    \forall n: \nat @ \\
\t1     square(n) = n * n
\end{axdef}
\end{demo}
Incidentally, this example illustrates that predicates and
declarations can be split between lines before or after any infix
symbol (such as $@$ here).  The strange hint \verb/\t1/\index{|\t|}
makes the corresponding line in the output have one helping of
indentation.  As things get more nested, you can say \verb/\t2/,
\verb/\t3/, and so on; these commands are ignored by the type
checker, so you are free to use them as you like to improve the look
of your specification.  The \verb/\t/ command usurps the name used by
\LaTeX\ for a tie-after accent\index{tie-after accent (\tie{oo})},
as in `\tie{oo}', so the \verb/fuzz/ style option renames the accent
as~\verb/\tie/.

Like schemas, axiomatic descriptions may omit the horizontal
dividing line and the predicates below it, leaving only the
declaration of some variables, as in the following example:
\begin{demo}
\begin{axdef}
    n\_disks: \nat
\end{axdef}
\gives
\begin{axdef}
    n\_disks: \nat
\end{axdef}
\end{demo}

\paragraph{Generic definitions%
        \ifmanual\ \rm (\ZRM\ Section~\protect\ref{ss:genconst})\fi}%
	\index{generic definition}
These are printed with the \verb/gendef/ environment:%
\index{|gendef| environment}
\begin{demo}
\begin{gendef}[X,Y]
    fst: X \cross Y \fun X
\where
    \forall x: X; y: Y @ \\
\t1     fst(x,y) = x
\end{gendef}
\gives
\begin{gendef}[X,Y]
    fst: X \cross Y \fun X
\where
    \forall x: X; y: Y @ \\
\t1     fst(x,y) = x
\end{gendef}
\end{demo} 
In this environment, the formal generic parameters%
\index{generic parameter} are an optional argument.  Omitting this
argument results in a box with a solid double bar at the top, which
can be used for uniquely defined constants that have no parameters.

\paragraph{Unboxed paragraphs}
Some Z paragraphs do not appear in boxes, and for these the
\verb/zed/ environment\index{|zed| environment} is used.  They may
be basic type definitions\index{basic type definition}%
\ifmanual\ (ZRM section~\ref{ss:basictype})\fi,
predicates written as global constraints\index{constraint}%
\ifmanual\ (Section~\ref{ss:constraint})\fi,
`horizontal' schema definitions\index{schema definition: horizontal}%
\ifmanual\ (Section~\ref{ss:schemadef})\fi,
abbreviation definitions\index{abbreviation definition}%
\ifmanual\ (Sections~\ref{ss:abbrev} and~\ref{ss:genconst})\fi,
or free type definitions\index{free type definition}%
\ifmanual\ (Section~\ref{s:freetype})\fi.
Here are a few examples:
\begin{demo}
\begin{zed}
    [NAME, DATE]
\also
    REPORT ::= ok | unknown \ldata NAME \rdata
\also
    \exists n: NAME @ \\
\t1     birthday(n) \in December.
\end{zed}
\gives
\begin{zed}
    [NAME, DATE]
\also
    REPORT ::= ok | unknown \ldata NAME \rdata
\also
    \exists n: NAME @ \\
\t1     birthday(n) \in December.
\end{zed}
\end{demo}
As the example illustrates, a full stop%
\index{full stop: allowed before |\end|} or comma is allowed just
before the closing \verb/\end/ command of any of the Z environments,
if that suits your taste (or is forced on you by a publisher's house
rules).  This punctuation is ignored by the type checker.

For large free type definitions\index{free type definition}, the
\verb/syntax/ environment 
provides a useful alternative to the \verb/zed/ environment, as the
following example suggests:\index{|syntax| environment}
\begin{demo}
\begin{syntax}
    OP  & ::= & plus | minus | times | divide
\also
    EXP & ::= & const \ldata \nat \rdata \\
        &  |  & binop \ldata OP \cross EXP 
                                   \cross EXP \rdata
\end{syntax}
\gives
\begin{syntax}
    OP  & ::= & plus | minus | times | divide
\also
    EXP & ::= & const \ldata \nat \rdata \\
        &  |  & binop \ldata OP \cross EXP \cross EXP \rdata
\end{syntax}
\end{demo}
Just as in the \verb/eqnarray/ environment of \LaTeX, the fields are
separated by \verb/&/ characters%
\index{|&| (alignment tab): ignored by type checker}; these are
ignored by the type checker. 

\section{Inside the boxes}\label{symtabs}

The first thing to notice about the text inside the boxes is that
multi-character identifiers\index{multi-character identifier} look
better than they do with ordinary
\LaTeX: instead of $\mit effective\ specifications$, you get
$effective\ specifications$.  The letters are not spread
apart, and ligatures like $ff$ and $fi$ are used.  This improvement is
achieved by an adjustment to the way \TeX\ treats letters in
mathematical formulas, and no special commands are needed in the
input file. 

Embedded underscore characters in identifiers can be printed with the
\verb/\_/ command\index{|\_| (underscore)}%
\index{underscore: in identifier}, so that \verb/not\_known/ prints
as $not\_known$.  The same command can be 
used for the dummy arguments\index{dummy argument (|\_|)} that
surround infix operators when they
appear on their own; for example, \verb/\_ + \_/ prints as $\_ +
\_~$.  Occasionally, the \verb/\_/ command makes an odd-looking
space in an identifier, as in $DISK\_POOL$, where the large
`overhang' of the italic $K$ has pushed the underline too far away.
This is an unfortunate consequence of \TeX's rules for setting
mathematical formulas, but it can be avoided by typing
\verb/"DISK\_POOL"/ instead (using double quote marks).  In Z text,
this produces a better-looking $"DISK\_POOL"$; the quote marks are
completely ignored by the \fuzz\ type checker.

\looseness=1
So much for the identifiers in Z specifications; what about the
mathematical symbols?  As in ordinary \LaTeX, they are typed using
\LaTeX\ commands with (hopefully) mnemonic names.  Many Z symbols
have the same name that they are given by \LaTeX, but some are given
a shorter name that is more suggestive of their use in Z.  The
\verb/fuzz/ style option also adjusts the way that many of the
symbols interact with \TeX's rules for spacing in
formulas\index{spacing rules of \TeX}, so that
they will look better in Z \hbox{specifications}.

A few symbols have two names, reflecting two different uses for the
symbol in Z:
\begin{itemize}
\item The symbol $\semi$ is called 
\verb/\semi/\index{|\semi|: vs.\ |\comp|}\index{|\semi| ($\semi$)!}
when it is used as an operation on schemas, and 
\verb/\comp/\index{|\comp|: vs.\ |\semi|}\index{|\comp| ($\comp$)!}
when it is used for composition of relations.
\item The symbol $\hide$ is called
\verb/\hide/\index{|\hide|: vs.\ |\setminus|}\index{|\hide| ($\hide$)!}
as the hiding operator of the schema calculus, and 
\verb/\setminus/\index{|\setminus|: vs.\ |\hide|}%
\index{|\setminus| ($\setminus$)!} as the set difference operator. 
\item The symbol $\project$ is called
\verb/\project/\index{|\project|: vs.\ |\filter|}%
\index{|\project| ($\project$)!} as the schema projection operator, and 
\verb/\filter/\index{|\filter|: vs.\ |\project|}%
\index{|\filter| ($\filter$)!} as the filtering operator on sequences.
\end{itemize}
Although the printed appearance of
each of these pairs of symbols is the same, the type checker
recognizes each member of the pair only in the appropriate
\hbox{context}.

The rest of this section contains lists of the commands defined in
the \verb/fuzz/ style option for printing the mathematical symbols
of Z\index{standard symbols+}.  The lists are organized according to
the class of symbol, with all relation signs in one list, all binary
operators in another, and so on.  This classification of standard
symbols is known to the \fuzz\ type checker, which uses the
information in parsing pieces of Z text. The symbols are also listed
according to subject on the Z reference card, part of the \fuzz\
package, with all the symbols connected with sets listed together,
and so~on.

Some of the symbols listed here look like words in a special
font:\index{font: significance to type-checker}
for example, \verb/\dom/ yields the word `$\dom$' in roman type,
\verb/\IF/ yields `$\bf if$' in bold-face type, and \verb/\prefix/
yields `$\prefix$' in sans-serif type.  As far as the type checker is
concerned, these symbols are completely different from plain
identifiers like $dom$, $if$, and~$prefix$.

\paragraph{Language elements}
Some symbols are basic elements of the Z language.  Here are the
mnemonics for them:\index{basic symbols}
\begin{symtab}
        \power & \verb/\power/ \\
        \cross & \verb/\cross/ \\
        = & \verb/=/ \\
        \in & \verb/\in/ \\
        | & \verb/|/ or \verb/\mid/ \\
        @ & \verb/@/ or \verb/\spot/ \\
        \theta & \verb/\theta/ \\
        \lambda & \verb/\lambda/ \\
        \mu & \verb/\mu/ \\
        \LET & \verb/\LET/ \\
        \Delta & \verb/\Delta/ \\
        \Xi & \verb/\Xi/ \\
        \defs & \verb/\defs/ \\
        \rlap{$\bf if\;then\;else$} \cr
                & \verb/\IF \THEN \ELSE/ \\
\end{symtab}
The commands \verb/|/ and \verb/@/ produce the vertical bar and spot
that appear in formulas like
\[ \forall x: \nat | x > 0 @ x \geq 1. \] 
They are typed as single characters, but like the other commands for
mathematical symbols, they may only be used in math mode.  For
upward compatibility, the commands \verb/\mid/ and \verb/\spot/ are
also provided.\index{upward compatibility}

The symbol $\bind$ that the \ZRM\ uses to write down bindings isn't
really part of Z, but it has a mnemonic anyway:
\begin{symtab}
        \bind & \verb/\bind/
\end{symtab}

\paragraph{Connectives and quantifiers}
Here are the commands for operators of propositional and predicate
logic and the schema calculus:\index{logical symbols}
\begin{symtab}
        \lnot & \verb/\lnot/ \\
        \land & \verb/\land/ \\
        \lor & \verb/\lor/ \\
        \implies & \verb/\implies/ \\
        \iff & \verb/\iff/ \\
        \forall & \verb/\forall/ \\
        \exists & \verb/\exists/ \\
        \exists_1 & \verb/\exists_1/ \\
        \hide & \verb/\hide/ \\
        \project & \verb/\project/ \\
        \pre & \verb/\pre/ \\
        \semi & \verb/\semi/
\end{symtab}

\paragraph{Fancy brackets}
Next come commands for the various sorts of fancy
brackets:\index{bracket symbols}
\begin{symtab}
        \{\;\} & \verb/\{  \}/ \\
        \langle\;\rangle & \verb/\langle  \rangle/ \\
        \lbag\;\rbag & \verb/\lbag  \rbag/ \\
        \ldata\;\rdata & \verb/\ldata  \rdata/ \\
        \limg\;\rimg & \verb/\limg  \rimg/
\end{symtab} 
In addition to the brackets used in specifications, there are the
special brackets used in the \ZRM\ to write down schema types. Like
$\bind$, they aren't part of Z, but they are sometimes useful in
writing {\em about\/} a \hbox{specification:}
\begin{symtab}
        \lblot\;\rblot & \verb/\lblot \rblot/
\end{symtab}

Those are all the symbols `built-in' to the Z language; now for the
symbols that are defined as part of the mathematical tool-kit. 

\paragraph{Constants and functions}
Some symbols name ordinary mathematical constants and
functions:\index{constant symbols}
\begin{symtab}
        \emptyset & \verb/\emptyset/ \\
        \bigcup & \verb/\bigcup/ \\
        \bigcap & \verb/\bigcap/ \\
        \dom & \verb/\dom/ \\
        \ran & \verb/\ran/ \\
        \nat & \verb/\nat/ \\
        \num & \verb/\num/ \\
        \nat_1 & \verb/\nat_1/ \\
        \# & \verb/\#/ \\
        \dcat & \verb/\dcat/
\end{symtab} 
The control sequence \verb/\empty/ was used in place of
\verb/\emptyset/ in previous versions of the \verb/fuzz/ style
option, but this caused problems because it clashes with a macro of
plain \TeX.\index{|\empty|: vs.\ |\emptyset| ($\emptyset$)}%
\index{|\empty| ($\emptyset$)!}
If an existing document uses \verb/\empty/, you can
still print it by adding the definition
\begin{quote}
        \verb/\renewcommand{\empty}{\emptyset}/
\end{quote}
in the document preamble (or by changing the name with an editor).
Both names are recognized by the \fuzz\ type checker.

\paragraph{Infix function symbols}
Next, here are the infix function symbols (class {\sf In-Fun}), each shown
with its priority, which is a number from 1 to 6, with 6 giving the
tightest binding:\index{operator symbol+}
\begin{symtab}
        \mapsto & \verb/\mapsto/\quad\hfill 1 \\
        \upto & \verb/\upto/\quad\hfill 2 \\
        + & \verb/+/\quad\hfill 3 \\
        - & \verb/-/\quad\hfill 3 \\
        \cup & \verb/\cup/\quad\hfill 3 \\
        \setminus & \verb/\setminus/\quad\hfill 3 \\
        \cat & \verb/\cat/\quad\hfill 3 \\
        \uplus & \verb/\uplus/\quad\hfill 3 \\
        \uminus & \verb/\uminus/\quad\hfill 3 \\
        * & \verb/*/\quad\hfill 4 \\
        \div & \verb/\div/\quad\hfill 4 \\
        \mod & \verb/\mod/\quad\hfill 4 \\
        \cap & \verb/\cap/\quad\hfill 4 \\
        \extract & \verb/\extract/\quad\hfill 4 \\
        \filter & \verb/\filter/\quad\hfill 4 \\
        \comp & \verb/\comp/\quad\hfill 4 \\
        \circ & \verb/\circ/\quad\hfill 4 \\
        \otimes & \verb/\otimes/\quad\hfill 4 \\
        \oplus & \verb/\oplus/\quad\hfill 5 \\
        \bcount & \verb/\bcount/\quad\hfill 5 \\
        \dres & \verb/\dres/\quad\hfill 6 \\
        \rres & \verb/\rres/\quad\hfill 6 \\
        \ndres & \verb/\ndres/\quad\hfill 6 \\
        \nrres & \verb/\nrres/\quad\hfill 6
\end{symtab}
The priority of operators means nothing to \LaTeX, but it is used
by the \fuzz\ type checker when it analyses expressions.

\paragraph{Postfix function symbols}
The standard postfix function symbols (class {\sf Post-Fun}) produce
different kinds of superscripts:
\begin{symtab}
        \inv & \verb/\inv/ \\
        \star & \verb/\star/ \\
        \plus & \verb/\plus/ \\
        {}^{n} & \verb/^{n}/
\end{symtab}
For example, \verb/R \star/ is printed as $R \star$, and
\verb/R^{n}/ is printed as $R^{n}$\index{superscript}%
\index{|^| (superscript)}.  The type checker regards this second
formula as equivalent to $iter~n~R$, as explained 
\ifmanual 
  on page~\pageref{p:iter} of 
\else 
  in 
\fi 
the \ZRM; the braces are {\em not\/} optional, even when the
superscript is a single digit.  For upward compatibility, the form
\verb/R \bsup n \esup/ is also recognized by both the style option
and the type checker.%
\index{upward compatibility}\index{operator symbol-}

\paragraph{Infix relation symbols}
The infix relation symbols have class {\sf In-Rel}:%
\index{relation symbols+}
\begin{symtab}
        \neq & \verb/\neq/ \\
        \notin & \verb/\notin/ \\
        \subseteq & \verb/\subseteq/ \\
        \subset & \verb/\subset/ \\
        < & \verb/</ \\
        \leq & \verb/\leq/ \\
        \geq & \verb/\geq/ \\
        > & \verb/>/ \\
\changewidth
        \prefix & \verb/\prefix/ \\
        \suffix & \verb/\suffix/ \\
        \inseq & \verb/\inseq/ \\
        \inbag & \verb/\inbag/ \\
        \subbageq & \verb/\subbageq/ \\
        \partition & \verb/\partition/ \\
        \inrel{R} & \verb/\inrel{R}/ \\
\end{symtab}
Besides the fixed infix relation symbols, an ordinary identifier
may be used as an infix relation if it is
underlined;\index{underlined relation symbol} to get
$x \inrel{R} y$, the input is \verb/x \inrel{R} y/.\index{|\inrel|}

\paragraph{Prefix relation symbols}
There is only one standard prefix relation symbol (class {\sf
Pre-Rel}):\index{relation symbols-}
\begin{symtab}
        \disjoint & \verb/\disjoint/
\end{symtab}

\paragraph{Infix generic symbols}
The infix generic symbols are assigned the class {\sf In-Gen}.  The
standard ones are all more-or-less fancy arrows, including the
famous `dead fish'.  The arrows used by \fuzz\ are a little bigger
than the ones that come with \LaTeX:%
\index{generic symbols+}\index{arrows}
\begin{symtab}
        \rel & \verb/\rel/ \\
        \pfun & \verb/\pfun/ \\
        \fun & \verb/\fun/ \\
        \pinj & \verb/\pinj/ \\
        \inj & \verb/\inj/ \\
        \psurj & \verb/\psurj/ \\
        \surj & \verb/\surj/ \\
        \bij & \verb/\bij/ \\
        \ffun & \verb/\ffun/ \\
        \finj & \verb/\finj/
\end{symtab}

\paragraph{Prefix generic symbols}
Prefix generic symbols are assigned class {\sf Pre-Gen} by the
type checker. Here is the standard list:\index{generic symbols-}
\begin{symtab}
        \power_1 & \verb/\power_1/ \\
        \id & \verb/\id/ \\
        \finset & \verb/\finset/ \\
        \finset_1 & \verb/\finset_1/ \\
        \seq & \verb/\seq/ \\
        \seq_1 & \verb/\seq_1/ \\
        \iseq & \verb/\iseq/ \\
        \bag & \verb/\bag/
\end{symtab}

As far as \LaTeX\ is concerned, the infix function and generic
symbols are defined to be binary operators, so they are separated
from their arguments by a medium space%
\index{spacing rules of \TeX+}.  The infix relation symbols are
defined as relation symbols, so that \LaTeX\ inserts thick spaces
around them.  The prefix generic symbols and most of the ordinary
symbols that denote functions are defined as operator symbols, so
\LaTeX\ inserts a thin space between the symbol and its argument.
The spacing rules of \TeX\ work well enough that most Z
specifications can be printed without the need for manual adjustment
of spacing, but the next section points out some places where human
assistance is needed.\index{standard symbols-}

\section{Fine points}\label{finepts}

In math mode, which is used for type-setting the contents of Z boxes, 
\LaTeX\ ignores all space characters in the input file. The spaces which
appear between elements of a mathematical formula are determined by
\LaTeX\ itself, working from information about the symbols in the
formula. Although (as described above) this information has been
adjusted in the \fuzz\ style option to make Z texts look as balanced
as possible, there are one or two situations in which \LaTeX\ needs a
little help.

Special care is needed when function application is indicated by
juxtaposing two identifiers, as in the expression
$rev~words$,\index{function: separated from argument by |~|} which
should be typed as \verb/rev~words/.%
\index{|~| (thin space): between function and argument} Typing just 
\verb/rev words/ results in the output $rev words$, since \LaTeX\
ignores the space separating the two identifiers.  The \verb/fuzz/
style option extends the definition of \verb/~/ so that in a formula
it inserts the same amount of space as the \LaTeX\
\verb/\,/ command.  The type checker completely ignores both the
\verb/~/ character and the \LaTeX\ spacing commands%
\index{space commands: ignored by type checker}, except that it
issues a warning if it finds that one is missing between
two identifiers.  It is not necessary to separate 
symbols like \verb/\dom/ and \verb/\ran/ from their arguments with a
\verb/~/, because \LaTeX\ inserts the right amount of space
automatically. For example, the input \verb/\dom f/ produces 
`$\dom f$'.

It is good style also to insert small spaces inside the braces of a
set comprehension, as in this example:%
\index{|~| (thin space): in set comprehension}
\begin{demo}
\{~x: \nat | x \leq 10 @ x * x~\}
\gives*
\[\{~x: \nat | x \leq 10 @ x * x~\}\]
\end{demo}
This helps to distinguish it visually from a set display, which should
not have the space:
\begin{demo}
\{1, 2, 3\}
\gives*
\[\{1, 2, 3\}\]
\end{demo}
Of course, the space symbol \verb/~/ is ignored by the type checker,
which distinguishes set displays from comprehensions by looking at
their contents, so the use of \verb/~/ is purely a matter of
aesthetics.  It also looks better if you add small spaces inside the
square brackets of `horizontal' schema texts.

\LaTeX\ also needs help when a binary operator appears at the end of a
line, as in the following example:%
\index{empty operand (|{}|)}\index{|{}| (empty operand)}
\begin{demo}
\begin{zed}
    directory' = directory \cup {} \\
\t3                 \{new\_name? \mapsto new\_number?\}
\end{zed}
\gives
\begin{zed}
    directory' = directory \cup {} \\
\t3                 \{new\_name? \mapsto new\_number?\}
\end{zed}
\end{demo}
\LaTeX\ will not recognize \verb/\cup/ as a binary operator and insert
the correct space unless it is surrounded by two operands, so the
empty operand \verb/{}/ has been inserted: this is ignored by the
type checker. This problem affects only binary operators; relation
signs do not need to be surrounded by arguments to be recognized by
\LaTeX\index{spacing rules of \TeX-}.

\section{Bits and pieces}\label{bitsnbobs}

Specification documents often contain mathematical text which does not
form part of the formal specification proper. This section describes some
environments for setting various kinds of informal mathematics; they are
provided for convenience, and they are all ignored by the
type checker. Besides these environments for making displays, run-in
mathematics can be set with the usual \verb/math/ environment, or with
the commands \verb/$ ... $/ or \hbox{\verb/\( ... \)/}. All the Z
symbols listed in Section~\ref{symtabs} can be used with these
commands.

The simplest displays are produced by the commands
\verb/\[ ... \]/.\index{|\[ ... \]|} This form acts just
like \verb/\begin{zed} ... \end{zed}/, except that the contents are
ignored by the type checker; it can be used to state informal
theorems about a specification, or to quote a piece of mathematical
text for discussion. The \verb/\[ ... \]/ commands generalize the
standard \LaTeX\ ones, because the displayed material can be several
lines.  Note, however, that the contents are set in text style
rather than display style.  Here is an example:
\begin{demo}
\[
    \exists PhoneDB @ \\
\t1     known = \emptyset
\]       
\gives
\[
    \exists PhoneDB @ \\
\t1     known = \emptyset
\]       
\end{demo}

\def\schemastar{{\tt schema*}}
Sometimes it is nice to expand a complicated schema expression and
display the results in a schema box with no name.  Such boxes can be
printed using the \verb/schema*/ environment, like
this:\index{\schemastar\ environment}
\begin{demo}
\begin{schema*}
        x, y: \nat
\where
        x > y
\end{schema*}
\gives
\begin{schema*}
        x, y: \nat
\where
        x > y
\end{schema*}
\end{demo}
Like all these environments, the \verb/schema*/ environment is
ignored by the type checker, and contributes nothing to the picture
of the formal specification it is building.

Another kind of mathematical display is provided by the \verb/argue/
environment\index{|argue| environment}.  This is like the \verb/zed/
environment, but the separation between lines is increased a little,
and page breaks may occur between lines.  The intended use is for
arguments like this:
\begin{demo}
\begin{argue}
    S \dres (T \dres R) \\
\t1     = \id S \comp \id T \comp R \\
\t1     = \id (S \cap T) \comp R & law about $\id$ \\
\t1     = (S \cap T) \dres R.
\end{argue}
\gives
\begin{argue}
    S \dres (T \dres R) \\
\t1     = \id S \comp \id T \comp R \\
\t1     = \id (S \cap T) \comp R & law about $\id$ \\
\t1     = (S \cap T) \dres R.
\end{argue}
\end{demo}
When the left-hand side is long, I find this style better than the
\LaTeX\ \verb/eqnarray/ style, which wastes a lot of space. 
Each line can have an optional second field, delimited by an
\verb/&/ character: it can be used for a hint why the expression is
considered equal to the previous line.

Finally, there is the \verb/infrule/ environment, used for inference
rules:\index{|infrule| environment}
\begin{demo}
\begin{infrule}
    \Gamma \vdash P
\derive[x \notin freevars(\Gamma)]
    \Gamma \vdash \forall x @ P
\end{infrule}
\gives
\begin{infrule}
    \Gamma \vdash P
\derive[x \notin freevars(\Gamma)]
    \Gamma \vdash \forall x \spot P
\end{infrule}
\end{demo}
The horizontal line is generated by \verb/\derive/; the optional
argument is a side-condition of the rule.

\section{Style parameters}\label{params}

A few style parameters affect the way Z text is set out; they can be
changed at any time if your taste doesn't match mine.%
\index{style parameters}

\begin{description}
\item[\tt\string\zedindent]\index{|\zedindent|} The indentation for
        mathematical text.  By default, this is the same as
        \verb|\leftmargini|, the indentation used for list
        environments.
\item[\tt\string\zedleftsep]\index{|\zedleftsep|} The space between
        the vertical line on the left of schemas, etc., and the
        maths inside. The default is 1~em.
\item[\tt\string\zedtab]\index{|\zedtab|} The unit of indentation
        used by \verb|\t|. The default is 2~em.
\item[\tt\string\zedbar]\index{|\zedbar|} The length of the
        horizontal bar in the middle of a schema. The default is
        6~em.
\item[\tt\string\zedskip]\index{|\zedskip|} The vertical space
        inserted by \verb/\also/. By default, this is the same as
        that inserted by \verb/\medskip/.
\item[\tt\string\zedsize]\index{|\zedsize|} A size-changing command
        used for Z text.  It may be redefined by a command such as
        \verb/\renewcommand{\zedsize}{\small}/.  The default is for
        Z text to be the same size as surrounding prose.
\end{description}

\chapter{Checking Specifications}\label{tc}

The \fuzz\ type checker can be run with the command
\begin{quote}
        \tt fuzz \lopt-aqstv\ropt\ \lopt-p {\it prelude\/}\ropt\ 
                \lopt {\it file\/} \dots\ropt
\end{quote}
It reads the input files, which are Z specifications written as
described in Chapter~\ref{tex}, extracts the pieces of formal text,
and produces messages on the standard error stream describing any
errors it finds.  If no input files are listed on the command line,
\fuzz\ reads the standard input stream instead.  Various optional
flags can be given to make \fuzz\ output additional information or
to modify the processing it performs.

Before reading the input files, the type checker reads a prelude
file\index{prelude file} that contains, amongst other things, the
definitions from the mathematical tool-kit%
\index{mathematical tool-kit} in Chapter~\ref{c:library} of the
\ZRM.  This prelude is a 
text file, and apart from a few special commands it has the same
form as any other input file.

Next, \fuzz\ reads the input files in sequence, keeping
the definitions from each file as it processes the next one, so that
specifications in several files can be analysed, with each file
using notation introduced by the ones before it.  The \LaTeX\
commands \verb/\input{...}/\index{|\input|: ignored by type checker}
and \verb/\include{...}/\index{|\include|: ignored by type checker}
are completely 
ignored by \fuzz, but much the same effect can be obtained by
listing the files on the command line.  

The following flags can be given on the command line to modify the
processing \fuzz\ performs:\index{flags: on command line+}
\begin{list}{}{\def\makelabel#1{\tt #1\hfil}}
\item[-a]\index{|-a| flag} Disable the system of type
        abbreviations\index{type abbreviation}
        that is normally used for printing types in reports and error
        messages.  Further details of the system of type
        abbreviations are in Section~\ref{abbrev}.

\item[-p{\it\ file\/}]\index{|-p| flag} Use {\it file\/} in place of
        the standard prelude\index{prelude file}.  This flag is
        useful if there are local extensions to the mathematical
        library; you can append them to a copy of the usual prelude
        file, and use \verb/-p/ to have \fuzz\ read the extended
        prelude in place of the usual one.

\item[-q]\index{|-q| flag} Allow implicit quantifiers%
        \index{implicit quantifiers}.  Any undeclared identifier that
        appears in the 
        predicate part of a schema, axiomatic description, or
        generic definition is treated as if it were declared by a
        universal quantifier surrounding the predicate, with a type
        inferred from the context.

\item[-d]\index{|-d| flag} Dependency analysis%
	\index{dependency analysis}.  The paragraphs of Z text are
	topologically sorted before checking, so that (with a few
	restrictions), the paragraphs can appear in the document in
	an order that best suits exposition.

\item[-s]\index{|-s| flag} Syntax check only. This flag overrides
        the \verb/-t/ flag, and can be used with the \verb/-v/ flag
        to extract the paragraphs of specification from the input
        files without checking them for type errors.

\item[-t]\index{|-t| flag}\index{reports} Display on the standard
        output the types of all names defined globally in the
        specification.  See Section~\ref{reports} for more details.

\item[-v]\index{|-v| flag} Display each paragraph of Z text on the
        standard output as it is extracted from the input files. See
        Section~\ref{reports} for more details.

\item[-l]\index{|-l| flag} Echo each paragraph of Z text in a
 	Lisp-like syntax.
\end{list}

If no \verb/-p/\index{|-p| flag} flag is present, the type checker
looks for the prelude file\index{prelude file: search rules} in an
implementation-dependent way.  In UNIX versions of \fuzz, if there
is an environment variable called \verb/FUZZLIB/%
\index{|FUZZLIB|: environment variable}, its value is used as the
location of the 
prelude file; otherwise, the type checker looks in a place
determined when \fuzz\ is installed.  In the IBM PC version, the
type checker looks in all the directories on the search path for a
file named \verb/fuzzlib/, and uses that as the prelude
file.\index{flags: on command line-}

\section{Definition before use}

\Fuzz\ checks the input specification for conformance with the
language rules of Z described in the \ZRM. 
These rules require that each identifier or schema name is declared or
defined before it 
is first used\index{definition before use+}. 
\Fuzz\ relaxes this condition if the \verb/-d/ flag is specified on
the command line, by reading the whole specification, then re-ordering
the paragraphs of Z material so that they appear in an order that is
acceptable according to the rules.
It is still necessary that the specification can be re-ordered in this
way, so it is not permitted (for example) to have two paragraphs, each
of which refers to definitions made in the other.

Another kind of exception to the principle of definition before use is
made for the symbols $\Delta$ and $\Xi$\index{|\Delta| ($\Delta$)}%
\index{|\Xi| ($\Xi$)}. These
may appear only as the first character of a schema name such as
$\Delta State$ (typed as \verb/\Delta State/). If no schema with
this name has been defined before the first use of the name, the
standard definition is inserted by the type checker%
\index{default definition: of $\Delta$ and $\Xi$}: see, for example,
the small 
specification in Chapter~\ref{example}. This convention allows both
the style where $\Delta State$ is an abbreviation for $State \land
State'$, and the style where it may be defined explicitly to be
something else.  Of course, if the definition is left implicit, then
$State: Exp $ must be defined as a schema {\em before\/} the first use of
$\Delta State$.

Normally, it is an error for a Z specification to use any identifier
that has not been declared; but some specifications are made much
shorter by the convention that undeclared variables are
universally quantified in the largest surrounding
predicate\index{implicit quantifiers}.  This
allows definitions like this:
\begin{axdef}
        double: \nat \fun \nat
\where
        double(x) = 2 * x
\end{axdef}
which would otherwise need a universal quantifier for $x$.  The
\verb/-q/ flag\index{|-q| flag} is provided so that specifications
using this informal convention can still be checked for other
errors.\index{definition before use-}

\section{Reports}\label{reports}

In addition\index{reports+} to checking a specification for errors,
the type checker can produce two kinds of reports about the formal
text it finds.  One report is activated by the \verb/-v/
flag\index{|-v| flag}, and contains an {\sc ascii} representation of
each Z paragraph; the other is activated by \verb/-t/%
\index{|-t| flag}, and lists the type of every name that is globally
defined by 
the specification.  For example, the schema 
%% \begin{zed} [NAME, PHONE] \end{zed}
\begin{schema}{PhoneDB}
    known: \power NAME \\
    phone: NAME \pfun PHONE
\where
    known = \dom phone
\end{schema}
is represented like this in the \verb/-v/ report:
\begin{verbatim}
schema PhoneDB
    known: P NAME
    phone: NAME -+> PHONE
where
    known = dom phone
end
\end{verbatim}
This output attempts to show the internal form of the
schema in a readable way.  The schema box is represented by the
keywords \verb/schema/, \verb/where/ and \verb/end/, and 
symbols like $\pfun$ are built up from {\sc ascii} characters.
The \verb/-t/ flag generates a report like this from the $PhoneDB$
schema:
\begin{verbatim}
Schema PhoneDB
    known: P NAME
    phone: NAME -+> PHONE
End
\end{verbatim}
\Fuzz\ discards the predicate part of the schema after checking it
for errors, because it contains no information that is needed to
check the rest of the specification; so the predicate part does not
appear in the \verb/-t/ report either.  On the other hand, all
inclusions of one schema in another are expanded before the report
is produced, so you can see exactly what components each schema has.
Like the types displayed in error messages, the types shown in
\verb/-t/ reports use the system of type abbreviations described in
Section~\ref{abbrev} to make them shorter and more readable.  If you
want to see the `real' types, you can use the \verb/-a/ flag in
combination with \verb/-t/.  In the example, the type of $phone$
would then be shown as
\[ \power(NAME \cross PHONE) \]
instead of 
\[ NAME \pfun PHONE. \]
An axiomatic description like this:
\begin{axdef}
        u, v: \nat
\where
        2 * u + v < 10
\end{axdef}
gives a report like this with the \verb/-v/ flag:
\begin{verbatim}
axdef
    u, v: NN
where
    (2 * u) + v < 10
end
\end{verbatim}
One thing that makes the \verb/-v/ reports particularly useful is
that extra parentheses are used to clarify the binding power of
operators, as in the expression \verb/(2 * u) - v/ here.  We all
know that multiplication binds more tightly than addition, but if
\fuzz\ reports an error in your specification that you 
just can't find, sometimes the \verb/-v/ report will reveal that an
expression is begin parsed in an unexpected way.

The \verb/-t/ flag shows separately each global definition which
\fuzz\ stores as it checks the specification: for our example, the
output is
\begin{verbatim}
Var u: NN

Var v: NN
\end{verbatim}
This shows that two global variables, $u$ and $v$, have been
introduced, and both have type $\nat$. Again, the predicate relating
$u$ and $v$ has been checked for errors and discarded.

The types of generic constants are shown under \verb/-t/ with
numeric markers in place of the formal generic parameters.  For
example, the definition
\begin{gendef}[X, Y]
        fst: X \cross Y \fun X \\
        snd: X \cross Y \fun Y
\where
        \forall x: X; y: Y @ \\
\t1             fst(x,y) = x \land \\
\t1             snd(x,y) = y
\end{gendef}
produces this output under \verb/-t/:
\begin{verbatim}
Genconst fst[2] : @1 x @2 -+> @1

Genconst snd[2] : @1 x @2 -+> @2
\end{verbatim}
Here, the \verb/[2]/ means that $fst$ and $snd$ each have two generic
parameters, and the markers \verb/@1/ and \verb/@2/ represent the
first and the second parameters respectively.  This form of output
is also used for generic constants defined with
$==$.\index{reports-}

\chapter{Advanced Features}\label{advanced}

%\looseness=1 %PAGE
For simple specifications, the features described in
Chapter~\ref{tc} are enough.  But more complex specifications may
need additional features of the type checker: they may add to Z's
standard list of infix operators, they may introduce new
mathematical concepts that can be reflected in the way \fuzz\
computes types for expressions, or they may be presented in an
informal way that requires slight adjustments to the strict
principle of definition before use.  All these features of
the type checker are activated by {\em directives}, special
comments that are processed by the type checker, but are ignored by
\LaTeX\ when it formats the specification for printing.

\section{Directives}\label{directives}

All the directives\index{directives+} that activate special features
of \fuzz\ start with the characters \verb/%%/, so \LaTeX\ sees them
as comments, and they have no effect on the printed version of the
specification.  Several of them introduce infix function or relation
symbols, and so on; others introduce new type abbreviations or mark
text to be ignored in type-checking.  A special directive allows
text to be included for type-checking but ignored by \LaTeX. Each
directive must appear at the beginning of a line in the input file,
and it continues to the end of that~line. Any other \TeX\
comment\index{comment: ignored by type checker} --
beginning with only one \verb/%/ or not appearing at the start of a
line -- is ignored by the type checker.

Here is a list of all the directives, with a brief explanation of
their meanings. In this list, $symbols$ stands for a sequence of
one or more identifiers or special symbols, separated by spaces.
\begin{list}{}{\def\makelabel#1{\tt #1\hfil}%
        \def\\{\leavevmode\hfil\break}}
\item[\%\%inop $symbols$ $n$]\index{|%%inop| directive}\\
        The symbols are introduced as infix function symbols with
        priority $n$, a digit in the range 1 up to 6.
\item[\%\%postop $symbols$]\index{|%%postop| directive}\\
        The symbols are introduced as postfix function symbols. 
\item[\%\%inrel $symbols$]\index{|%%inrel| directive}\\
        The symbols are introduced as infix relation symbols.
\item[\%\%prerel $symbols$]\index{|%%prerel| directive}\\
        The symbols are introduced as prefix relation symbols.
\item[\%\%ingen $symbols$]\index{|%%ingen| directive}\\
        The symbols are introduced as infix generic symbols.
\item[\%\%pregen $symbols$]\index{|%%pregen| directive}\\
        The symbols are introduced as prefix generic symbols.
\item[\%\%type $symbols$]\index{|%%type| directive}\\
        Each of the symbols must have a previous global definition as a
        variable or generic constant with a set type, or as a
        schema; they are made into type abbreviations.  The
        definitions of the symbols must be monotonic, as described
        in Section~\ref{abbrev}.
\item[\%\%tame $symbols$]\index{|%%tame| directive}\\
        Each of the symbols must have a previous global definition 
        as a generic function; they 
        are marked as tame functions in the sense explained in
        Section~\ref{abbrev}.   
\item[\%\%unchecked]\index{|%%unchecked| directive}\\
        The immediately following Z environment is completely
        ignored by the type checker.
\item[\%\% $text$]\index{|%%| directive}\\
        The string $text$ is processed by the type checker, but
        ignored in printing the specification.
\end{list}
The directives which introduce infix function symbols, etc., override
any previous directive for the same symbol.  There are a few other
directives not listed here that are used in the standard prelude to
set up the type checker's symbol tables.  They may not be used in
ordinary specification documents.\index{directives-}
 
\section{User-defined operators}\label{userops}

Specifications\index{operator symbol: user-defined+} that use Z's
tool-kit of mathematical symbols are made easier to read by the fact
that many of these symbols are defined as infix operators of one
kind or another.  The \ZRM\ lists the standard operator symbols on
page~\pageref{p:stdsym}, but does not prescribe any way of adding to
this list.  With \fuzz, the standard set of operators may be
extended using the directives listed in Section~\ref{directives}.

There are three stages in adding a new operator symbol to a
specification.  The first is to write a definition that tells
\LaTeX\ how to print the symbol, choosing a command name to
represent the symbol in the input file.  The second stage is to
announce to \fuzz\ (using one of the directives) that the symbol is
to be used as an infix operator of a specified kind; this
announcement is often best complemented by an explanation for the
human reader of the relative binding power of the operator.  The
third stage in adding a new infix symbol is to write its
mathematical definition in Z.

To understand this procedure, it may help to think of three
potential `audiences' for the specification document: \LaTeX, the
\fuzz\ type checker, and the human reader.  The specification is
written as an input file for \LaTeX\, using a certain command to
represent the symbol.  To format the specification, the \LaTeX\
program must know what printed symbol to associate with the
command.  This association is set up in the first stage of
introducing the symbol, by defining the \LaTeX\ command with
\verb/\newcommand/.

\looseness=-1
The type checker does not need to know how the symbol is to appear
on the page, and it ignores the \LaTeX\ definition that contains
this information.  But it does need to know whether it is an infix
function symbol, a prefix generic symbol, and so on.  This
information is given by a directive in the second stage; this
directive is seen by \LaTeX\ as a comment, so it does not affect the
printed appearance of the specification.

Human readers want to be told how to read formulas that use the
symbol, information that is best conveyed by a small example given
in stage 2.  They also want to see its definition, given in
stage 3.  The fact that the definition for \LaTeX\ in stage 1 and
the directive for \fuzz\ in stage 2 are invisible in the printed
text allows the specification author to explain the significance of
the new symbol appropriately.

Here is an example which shows how to introduce a new infix function
symbol. Let's define $\diamond$ to be an operator which takes two
sequences and concatenates the reverse of the first with the
second.  Luckily, there is already a \LaTeX\ command \verb/\diamond/
for $\diamond$, so we don't need to bother with the first stage.
For the second stage, we announce in a directive that
\verb/\diamond/ is an infix function symbol, choosing its priority
to be 3:\index{|%%inop| directive}
\begin{quote}
        \verb/%%inop \diamond 3/
\end{quote}
It is important to announce \verb/\diamond/ as an infix symbol {\em
before\/} giving its definition, or the type checker will not be
able to parse the definition correctly. The new symbol should be
introduced for the human reader with a suitable English sentence. In this
case, one might say ``The symbol $\diamond$ will be used as an infix
function symbol; it has the same binding power as $\cat$.''
Personally, I like to avoid using numeric priorities to explain the
binding power of infix symbols: if you exploit relative binding
powers, it's a good idea to give an example, such as ``The
expression $a \filter S \diamond b$ should be read as $(a \filter S)
\diamond b$.''

Now for the third stage: we define the operator with a generic
definition like this:\index{|gendef| environment}
\begin{demo}
\begin{gendef}[X]
    \_ \diamond \_: \seq X \cross \seq X \fun \seq X
\where
    \forall a, b: \seq X @ \\
\t1     a \diamond b = (rev~a) \cat b
\end{gendef}
\gives
\begin{gendef}[X]
    \_ \diamond \_: \seq X \cross \seq X \fun \seq X
\where
    \forall a, b: \seq X @ \\
\t1     a \diamond b = (rev~a) \cat b
\end{gendef}
\end{demo}
The `quoted' name $\_ \diamond \_$ is used in the declaration,
because $\diamond$ is now an infix operator. After defining
$\diamond$, we can begin to use it in expressions, like this:
\begin{demo}
\begin{zed}
    \langle 1, 2, 3 \rangle 
    \diamond \langle 4, 5, 6 \rangle
        = \langle 3, 2, 1, 4, 5, 6 \rangle.
\end{zed}
\gives
\begin{zed}
    \langle 1, 2, 3 \rangle \diamond \langle 4, 5, 6 \rangle
        = \langle 3, 2, 1, 4, 5, 6 \rangle.
\end{zed}
\end{demo}

\def\subseq{\mathrel{\sf subseq}} 
As another example, let's define $\subseq$ as the relation which
holds between two sequences when the all the elements of the first
one also appear in the second one in the same order.  The first
stage is to define a \LaTeX\ command \verb/\subseq/ which produces
the relation symbol $\subseq$ in the right style of type. This is
achieved by
\begin{quote}
        \verb/\newcommand{\subseq}{\mathrel{\sf subseq}}/
\end{quote}
\looseness=1
The \TeX\ primitive \verb/\mathrel/ is not documented in the \LaTeX\
manual, but it causes its argument to be considered as a relation
symbol in mathematical formulas, so that the right amount of space
will be inserted around it. There is also \verb/\mathbin/ for binary
operators, and others for brackets, prefix operators like `$\dom$'
and so on: see the {\TeX}book, page~155.  The font-changing command
\verb/\sf/ causes the word `{\sf subseq}' to appear in sans-serif
type, like the standard infix relation symbols in the~\ZRM.

The second stage is to announce to the type checker that \verb/\subseq/
is to be used as an infix relation symbol:\index{|%%inrel| directive}
\begin{quote}
        \verb/%%inrel \subseq/
\end{quote} 
Both \LaTeX\ and the type checker now know all they need
about our new symbol, and we can proceed to the third stage, giving
its mathematical definition:\vadjust{\break}
\begin{demo}
\begin{gendef}[X]
    \_ \subseq \_: \seq X \rel \seq X
\where
    \forall a, b: \seq X @ \\
\t1     a \subseq b \iff (\exists S: 
            \power \nat_1 @ a = S \extract b)
\end{gendef}
\gives
\begin{gendef}[X]
    \_ \subseq \_: \seq X \rel \seq X
\where
    \forall a, b: \seq X @ \\
\t1     a \subseq b \iff (\exists S: 
            \power \nat_1 @ a = S \extract b)
\end{gendef}
\end{demo}
The symbol `$\subseq$' can now be used in formulas:%
\index{operator symbol: user-defined-}
\begin{demo}
\begin{zed}
    \langle 2, 4 \rangle 
        \subseq \langle 1, 2, 3, 4 \rangle.
\end{zed}
\gives
\begin{zed}
    \langle 2, 4 \rangle 
        \subseq \langle 1, 2, 3, 4 \rangle.
\end{zed}
\end{demo}

The technique of defining a new \LaTeX\ command for an infix
operator can also be used with schemas%
\index{schema name: containing special characters}.  This is useful
because some
styles of specification encourage schema names like $\Phi State$
that do not conform to \fuzz's syntax for schema names (see
page~\pageref{lexis}).  The trick
is to define a new \LaTeX\ command like this:
\begin{verbatim}
\newcommand{\PhiState}{\Phi State}
\end{verbatim}
then define the schema like this:
\begin{verbatim}
\begin{schema}{\PhiState} ...
\end{verbatim}
After these definitions, the command \verb/\PhiState/ prints as
$\Phi State$, and is recognized as a schema name by the type checker.

\newpage
\section{Type abbreviations}\label{abbrev}

The \fuzz\ type checker\index{type abbreviations+} separates its
work into two parts: computing the types of all the expressions
which appear in a specification, and matching types with each other
where they must agree. \Fuzz\ uses the type system described in the
\ZRM\ for comparing types, but it uses a richer type system when
computing the types of expressions. This makes for better error
messages, because types in the richer system are shorter and easier
to understand.  Because the type system used for matching types is
exactly that described in the \ZRM, none of the power or
expressiveness of Z is lost.

The \ZRM's type system has only three type constructors: power set
$\power$, Cartesian product $\cross$, and schema type
$\lblot\ldots\rblot$. In this system, even simple expressions can
have rather complicated types: for example, $\dom[X,Y]$ has the type
\[
        \power(\power(X \cross Y) \cross \power X).
\]
This is rather difficult to read, and the fact that $\dom$ is a
function has been lost.  For the same expression, \fuzz\ computes the
type
\[
        (X \rel Y) \pfun \power X,
\]
which is shorter, and closer to the `type' given to $\dom$ in its
declaration. This has been achieved by using the {\em type
abbreviations\/} `$\rel$' and `$\pfun$'. 

We can recover the real type of $\dom[X,Y]$ by expanding the
abbreviations\index{type abbreviations: expansion}, replacing both
$A \rel B$ and $A \pfun B$ by $\power(A \cross B)$. The value of $A
\rel B$ is actually equal to this set, but $A \pfun B$ is a proper
subset of $\power(A \cross B)$, so the type we obtain by expansion
is larger as a set than the unexpanded type. This means that
information has been lost in the expansion, but it is safe to assume
that any object in the original, abbreviated type is also in the
larger, expanded type.

The algorithm used by \fuzz\ to check that types agree across
equality and membership signs and in function applications is
equivalent to expanding all abbreviations and checking for an exact
match, so you are still free to exploit the way notions like
function and sequence are defined in Z: where convenient, you can
use the fact that functions are sets of ordered pairs, and sequences
are functions defined on a segment of the integers.

Here is a list of the type abbreviations defined in the standard
prelude:
\begin{syntax}
        \nat & \subseteq & \num \\
        X \rel Y & = & \power(X \cross Y) \\
        X \pfun Y & \subseteq & \power(X \cross Y) \\
        X \ffun Y & \subseteq & \power(X \cross Y) \\
        \finset X & \subseteq & \power X \\
        \seq X & \subseteq & \power(\nat \cross X) \\
        \bag X & \subseteq & \power(X \cross \nat)
\end{syntax}
A notable absence from this list is the total function arrow
$\fun$.  It is omitted because it is not {\em
monotonic\/}\index{monotonicity: of type abbreviations} like the
others\index{total functions ($\fun$): not monotonic}: if $S
\subseteq T$, then it is not generally true that $(S \fun V)
\subseteq (T \fun V)$.  For example, the set $even \fun \num$
contains the functions from numbers to numbers that are
defined exactly when the argument is in the set $even$; one of its
members is the function $half$ that takes an even argument and
halves it.  This set is not a subset of $\num \fun \num$, which
contains only functions that are defined on the whole of $\num$; in
fact, the two sets of functions are disjoint.

It is important that abbreviations are monotonic, because otherwise
the method \fuzz\ uses to calculate types may conclude that an
expression has a certain type, when in fact the value of the
expression is not a member of the type.  The property of being a
total function is simply not one that can be expressed in the type
system.

The way that the type checker uses the two-level type system is
illustrated by the predicate
\[ phone' = phone \oplus \{name? \mapsto number?\}. \]
The type checker calculates the types in this equation as shown in
the following listing, where each line shows a sub-expression and
its type:
\begingroup\interzedlinepenalty=\interdisplaylinepenalty
%%unchecked
\begin{syntax}
        phone & : & NAME \pfun PHONE \\
        name? & : & NAME \\
        number? & : & PHONE \\
        name? \mapsto number? & : & NAME \cross PHONE \\
        \{name? \mapsto number?\} & : & \power(NAME \cross PHONE) \\
        \_ \oplus \_ & : & (X \rel Y) \cross (X \rel Y) 
                                                \pfun (X \rel Y)
\end{syntax}
\endgroup
So far, so good.  Now the type checker finds that the operator
$\oplus$, which expects two relations as its arguments, is being
applied to arguments that are a partial function and a set of
ordered pairs.  This is allowed, for although they are
superficially different, all three types
\[
        NAME \rel PHONE \\
        NAME \pfun PHONE \\
        \power(NAME \cross PHONE) 
\]
can be changed into the third one by expanding abbreviations.  So
the type checker ascribes to the right-hand side the type $NAME \rel
PHONE$, the result type of $\oplus$ with $NAME$ substituted for the
generic parameter $X$ and $PHONE$ substituted for $Y$.
The left-hand side of the equation has type $NAME \pfun PHONE$.
Again, this is superficially different from the type ascribed to the
right-hand side, but they have a common expansion, so the equation
is accepted.

New type abbreviations can be introduced using the
\verb/%%type/\index{|%%type| directive}
directive (see Section~\ref{directives}). Any symbol defined globally
in the specification as having a set type can be marked as a type
abbreviation, but it is up to the user to check that the requirement
is satisfied that it be monotonic, because there is no way to check this
mechanically. If a type abbreviation is not monotonic, no spurious
error messages will be produced, but the types displayed in error
messages may be misleading.

Another concept related to type abbreviations is that of a {\em
tame\/} function\index{tame functions+}.  An example is the
concatenation operator $\cat$%
\index{concatenation ($\cat$): as tame function} on sequences.  The
enriched type 
system gives this operator the type
\[ \seq X \cross \seq X \pfun \seq X, \] 
and allows the type checker to deduce unaided that applying this
operator to two sequences $s$ and $t$ of the same type $\seq X$
yields another {\em sequence\/} $s \cat t$, and not simply an
element of the expanded type $\power(\num \cross X)$.  But there
are other relevant properties of $\cat$ that cannot be deduced from
its enriched type.  One such property is that concatenating two elements of
$\seq \nat$ gives another element of $\seq \nat$, and not simply an
element of the type $\seq \num$.  This is a consequence of the fact
that $\cat$ is tame, meaning that for any set $S$, if $s$ and $t$
are in $\seq S$, so is $s \cat t$.  Another application of the
tameness of $\cat$ is the observation that concatenating two
sequences of sequences gives another sequence of {\em sequences},
and not just a sequence of relations.

Almost all the generic functions defined in the standard library are
tame, but one that is not tame is the reflexive--transitive
closure operator $\_\star$%
\index{reflexive--transitive closure ($\_\star$): not a tame function}.
Even though $succ$ is in $\nat \rel 
\nat$ according to its \ZRM\ definition, $succ\star$ is not in $\nat
\rel \nat$ but is in $\num \rel \num$.

In order to allow the type checker to deduce reasonable types for
expressions involving tame functions, while treating non-tame
functions correctly, there is a directive
\verb/%%tame/\index{|%%tame| directive} for announcing that a
generic function is tame for the types with which it was declared
(for details, see Section~\ref{directives}).  Like the \verb/%%type/
directive, it is used extensively in the standard prelude, but is
less commonly used in ordinary specification documents.  In checking
a function application, the type checker treats it as a special case
if the function being applied is known to be tame.  If the function
being applied is not a generic function symbol with implicit
parameters that has been announced as tame, the type checker uses a
more conservative method of calculating the type.%
\index{type abbreviations-}\index{tame functions-}

\section{Invisible and unchecked paragraphs}\label{unchecked}

The \verb/%%unchecked/\index{|%%unchecked| directive} directive
causes the next use of a Z environment -- one of \verb/zed/,
\verb/axdef/, \verb/schema/ or \verb/gendef/ -- to be ignored by the
type checker.  It is useful for schemas which contain informal
nonsense, as in the following example:
\begin{demo}
%%unchecked
\begin{schema}{Register}
    enrolled: \power STUDENT \\
    completed: \power STUDENT
\where
    \mbox{\dots\ Invariant \dots}
\end{schema}
\gives
%%unchecked
\begin{schema}{Register}
    enrolled: \power STUDENT \\
    completed: \power STUDENT
\where
    \mbox{\dots\ Invariant \dots}
\end{schema}
\end{demo}
A specification document might contain an informal definition of the
$Register$ schema like this one, followed by a discussion of what
invariant might be appropriate, and finally the proper definition of
$Register$.  By using the \verb/%%unchecked/ directive, the writer
can have both schema boxes printed like any other, but have the
type checker take notice of only the second, complete one.

A sort of opposite to \verb/%%unchecked/ is the \verb/%%/
directive.\index{|%%| directive}
This is useful for including short paragraphs that are processed by
the type checker but should not appear in the output:
\begin{demo}
We give the name $FN$ 
%% \begin{zed} [FN] \end{zed}
to the set of all file names.
\gives
\begin{quote}
We give the name $FN$ 
%% \begin{zed} [FN] \end{zed}
to the set of all file names.
\end{quote}
\end{demo}
Here the author wanted to avoid cluttering up the specification as
printed with the formal definition of $FN$, but needed to include
the definition for the sake of the type checker.

A second use for \verb/%%/ is in combination with \verb/%%unchecked/.
In the $Register$ example, the whole schema is ignored by the
type checker, and this may cause problems if a complete definition
is not included later, and the declaration part of
the schema is needed for type-checking the rest of the specification.
The best thing in this case is to include a brief definition of
$Register$ using \verb/%%/:
\begin{verbatim}
%% \begin{schema}{Register}
%%     enrolled, completed: \power STUDENT
%% \end{schema}
\end{verbatim}
This can be put next to the unchecked but visible definition.
This technique is sometimes useful when it seems clearer to present
the parts of a specification in a different order from the one that
would be needed to follow strictly the principle of definition
before use.\index{definition before use}

Another use for the \verb/%%/ directive is to introduce local
variables for a predicate:
\begin{demo}
A name $n$ is included in the output exactly if
\begin{zed}
%% \exists BirthdayBook; n: NAME; today: DATE @
    birthday(n) = today
\end{zed}
\gives
\begin{quote}
A name $n$ is included in the output exactly if
\begin{zed}
%% \exists BirthdayBook; n: NAME; today: DATE @
    birthday(n) = today
\end{zed}
\end{quote}
\end{demo}
The existential quantifier is included only to introduce the variables
used in the predicate; it is used for its effect of declaring some
variables rather than for its logical meaning. In such a simple
example as this one, it would be easier to exclude the predicate from
checking, either by marking it with \verb/%%unchecked/, or by using 
\verb/\[ ... \]/ in place of the \verb/zed/ environment. In more
complicated examples, however, this technique allows more thorough
type-checking to be done.

\chapter{Error Messages}\label{mess}

Most of the messages about type errors produced by \fuzz\ come with
supporting evidence about the expressions and types involved.  These
expressions and types are obtained from the internal data structures
that \fuzz\ uses to represent them, so they may differ slightly from
the actual forms written in the specification.  For example,
expressions tend to have extra brackets added to clarify the
precedence of operators. 

This corroborative detail makes the meaning of most error messages
clear; but here anyway is a list of the semantic error
messages \fuzz\ produces, together with a little more information
about each one.  The numbers in square brackets give the pages in
the \ZRM\ where relevant rules of the language are explained.
\begin{trivlist}
\item[]{\bf Application of a non-function}\\
        In an expression of the form $f(E)$, the sub-expression $f$
        does not have type $\power (t_1 \cross t_2)$ for some types
        $t_1$ and $t_2$. \zrmref{p:apply} 

\item[]{\bf Argument $k$ has wrong type}\\
        In an expression of the form $f(E_1, E_2, \ldots, E_n)$, the
        type of $E_k$ does not match the type expected for the $k$'th
        argument of $f$. Also applies to expressions of the form
        $E_1\:f\:E_2$, where $f$ is an infix function symbol.
        \zrmref{p:apply}

\item[]{\bf Argument $k$ of $\cross$ must be a set}\\
        In an expression $E_1 \cross E_2 \cross \cdots \cross E_n$,
        the sub-expression $E_k$ does not have type $\power t$ for
        some type $t$. \zrmref{p:Pcross}

\item[]{\bf Argument of application has wrong type}\\
        In an expression $f(E)$, the type of $E$ is different from the
        type expected for the argument of $f$. \zrmref{p:Pcross}

\item[]{\bf Argument of $\power$ must be a set}\\
        In an expression $\power E$, the sub-expression $E$ does not
        have type $\power t$ for some type $t$. \zrmref{p:Pcross}

\item[]{\bf Argument of postfix operator has wrong type}\\
        In an expression $E~\omega$, where $\omega$ is a postfix
        function symbol, the type of $E$ is not the type expected
        for the argument of $\omega$. \zrmref{p:operators}

\item[]{\bf Argument of prefix relation has wrong type}\\
        In a predicate $R\;E$, where $R$ is a prefix relation symbol,
        the type of $E$ is not the type expected for the argument of
        $R$. \zrmref{p:relations}

\item[]{\bf Argument of selection must have schema type}\\
        In an expression $E.x$, the sub-expression $E$ does not have a
        schema type. \zrmref{p:select}

\item[]{\bf Basic type $X$ cannot have parameters}\\
        A reference to the basic type $X$ has actual generic
        parameters. Only generic constants and schemas can have
        parameters. \zrmref{p:genuse}

\item[]{\bf Basic type name $X$ multiply declared}\\
        The basic type name $X$ appears more than once in the same
        basic type definition or formal generic parameter list. The
        second occurrence is ignored.
        \zrmref{ss:basictype,ss:genschema,ss:genconst}

\item[]{\bf Component $x$ has wrong type in schema reference}\\
        A schema reference is used as a predicate, and the type of
        the component $x$ in the schema is different from the type
        $x$ has in the current context.  \zrmref{p:srefpred}

\item[]{\bf Constructor name $x$ multiply declared}\\
        The constructor name $x$ appears in more than one branch of
        the same free type definition.  The second occurrence is
        ignored.  \zrmref{s:freetype}

\item[]{\bf Decoration ignored in schema reference}\\
        A schema reference used as an expression contains a non-empty
        decoration. The decoration is ignored. \zrmref{p:srefexp}

\item[]{\bf Function expects $n$ arguments}\\
        In an expression $f(E_1, E_2, \ldots, E_k)$, the number of
        arguments supplied, $k$, is different from the number $n$ expected
        by $f$. \zrmref{p:apply}

\item[]{\bf Generic constant $x$ expects $n$ parameters}\\
        In an expression $x[E_1, E_2, \ldots, E_k]$, the number of
        parameters supplied, $k$, is different from the number $n$ expected
        by $x$. Also applies to expressions of the form $E_1\;x\;E_2$
        or $x\;E$ where $x$ is an infix or prefix generic symbol.
        \zrmref{p:genuse}

\item[]{\bf Global name $x$ multiply declared}\\
        The name $x$ has more than one global definition. The second
        definition overrides the first. \zrmref{p:multidec}

\item[]{\bf Hiding non-existent component $x$}\\
        In a schema expression which hides certain components of a
        schema, the component $x$ being hidden does not exist.
        Applies to the quantifiers $\forall$, $\exists$ and
        $\exists_1$, and the explicit hiding operator $\hide$.
        \zrmref{p:hide}

\item[]{\bf Identifier $x$ is not declared}\\
        No declaration or definition of $x$ is in scope.
        \zrmref{s:scope}

\item[]{\bf Implicit parameters not completely determined}\\
        The actual parameters of a generic constant have been left
        implicit, but they are not completely determined by the
        context. Also applies to set, sequence and bag displays.
        \zrmref{p:undetvar}

\item[]{\bf Infix operator $\omega$ is not a function}\\ 
        In an expression $E_1 \mathbin{\omega} E_2$, where $\omega$
        is an infix function symbol, $\omega$ does not have the
        right type to be a function. \zrmref{p:operators}

\item[]{\bf Infix relation symbol $R$ is not a binary relation}\\
        In a predicate $E_1\;R\;E_2$, where $R$ is an infix relation
        symbol, $R$ does not have type $\power (t_1 \cross t_2)$ for
        some types $t_1$ and $t_2$.  \zrmref{p:relations}

\item[]{\bf Left argument of operator $\omega$ has wrong type}\\
        In an expression $E_1 \mathbin{\omega} E_2$, where $\omega$
        is an infix function symbol, the type of $E_1$ does not
        match the type expected for the left argument of $\omega$.
        \zrmref{p:operators}

\item[]{\bf Let variable $x$ multiply declared}\\
        In a let-expression $\LET x_1 == E_1; x_2 == E_2~\ldots$,
        the same identifier appears more that once among the
        left-hand sides $x_1$, $x_2$, \dots~.  \zrmref{p:letexp}

\item[]{\bf Name $x$ is not a schema}\\
        The name $x$ appears in a schema reference, but it is not
        defined as a schema. \zrmref{s:schemaref}

\item[]{\bf Postfix operator $\omega$ is not a function}\\
        In a predicate $E~\omega$, where $\omega$ is a postfix
        function symbol, $\omega$ does not have the right type to be a
        function. \zrmref{p:operators}

\item[]{\bf Prefix relation symbol $R$ is not a set}\\
        In a predicate $R\;E$, where $R$ is a prefix relation symbol,
        $R$ does not have a set type. \zrmref{p:relations}

\item[]{\bf Renamed component $x$ does not exist}\\
        In a schema reference $S[y/x, \dots]$, there is no component
        of $S$ with the name $x$. \zrmref{s:schemaref}

\item[]{\bf Renaming ignored in schema reference}\\
        A schema reference used as an expression contains a non-empty
        renaming part. The renaming part is ignored. \zrmref{p:srefexp}

\item[]{\bf Right argument of operator $\omega$ has wrong type}\\
        In an expression $E_1 \mathbin{\omega} E_2$, where $\omega$
        is an infix function symbol, the type of $E_2$ does not
        match the type expected for the right argument of $\omega$.
        \zrmref{p:operators}

\item[]{\bf Schema $S$ expects $n$ parameters}\\
        In a schema reference $S[E_1, E_2, \ldots, E_k]$, the number
        of actual generic parameters, $k$, is different from the number
        $n$ expected by $S$. \zrmref{ss:genschema}

\item[]{\bf Schema $S$ is not defined}\\
        The schema $S$ appearing in a schema reference has not been
        defined. \zrmref{s:schemaref} 

\item[]{\bf Selecting non-existent component $x$}\\
        In an expression $E.x$, the name $x$ is not one of the
        components in the type of $E$. \zrmref{p:select}

\item[]
{\bf Set-valued expression required in actual generic parameter}\\
{\bf Set-valued expression required in declaration}\\
{\bf Set-valued expression required in free type definition}\\
        An expression $E$ appearing as an actual generic parameter (as
        in $A[E]$), in a declaration (as in $x: E$), or in a free type
        definition -- as in 
        \[ T ::= \ldots | f \ldata E \rdata | \ldots \]
        -- does not have a set type.
        \zrmref{s:decl,ss:genschema,ss:genconst,s:freetype}

\item[]
{\bf Tame function $x$ has no global definition}\\
{\bf Tame function $x$ is not a generic function}\\
        The name $x$ appears in a \verb/%%tame/ directive, but
        either it has no global definition, or it is not a generic
        function.  The directive is ignored.

\item[]
{\bf Type abbreviation $x$ has no global definition}\\
{\bf Type abbreviation $x$ is not a set}\\
        The name $x$ appears in a \verb/%%type/ directive, but
        either it has no global definition, or it does not have a set
        type. The directive is ignored.

\item[]
{\bf Type mismatch in bag display}\\
{\bf Type mismatch in sequence display}\\
{\bf Type mismatch in set display}\\
        In a set display $\{E_1, E_2, \ldots, E_n\}$, the
        expressions $E_i$ do not all have the same type. Also applies
        to sequence and bag displays. \zrmref{p:tuple,p:displays}

\item[]{\bf Type mismatch in conditional expression}\\
        In a conditional expression $\IF P \THEN E_1 \ELSE E_2$, the
        types of $E_1$ and $E_2$ are different. \zrmref{p:condexp}

\item[]{\bf Type mismatch in declarations of $x$}\\
        In a declaration part, or a schema expression, there are two
        declarations of $x$ which give it different types.
        \zrmref{ss:combschema,p:combdecl}
        
\item[]{\bf Type mismatch in hiding variable $x$}\\
        In a quantified schema expression $\forall D | P @ S$, the
        type of component $x$ in the schema $S$ is different from the
        type it is given by the declaration $D$. Also applies to the
        quantifiers $\exists$ and $\exists_1$. \zrmref{p:hide}

\item[]
{\bf Type mismatch in left argument of infix relation}\\
{\bf Type mismatch in right argument of infix relation}\\
        In a predicate $E_1\;R\;E_2$, where $R$ is an infix relation
        symbol, either $E_1$ or $E_2$ has a type different from that
        expected by $R$. \zrmref{p:relations}

\item[]{\bf Type mismatch in piping}\\
        In a piping $S_1 \pipe S_2$, the type of some output $x!$ of
        $S_1$ is different from the type of the input $x?$ of
        $S_2$.  \zrmref{p:opcomp}

\item[]{\bf Type mismatch in sequential composition}\\
        In a sequential composition $S_1 \semi S_2$, the type of some
        component $x'$ in $S_1$ is different from the type of $x$ in
        $S_2$.  \zrmref{p:opcomp}

\item[]{\bf Types do not agree in equation}\\
        In a predicate $E_1 = E_2$, the left hand side has a different
        type from the right hand side.  \zrmref{p:eqmem}

\item[]{\bf Types do not agree in set membership}\\
        In a predicate $E_1 \in E_2$, the type of the right hand side
        is not the same as $\power t$, where $t$ is the type of the
        left hand side.  \zrmref{p:eqmem}

\item[]{\bf Variable $x$ cannot have parameters}\\
        A reference to the variable $x$ has actual generic
        parameters. Only generic constants and schemas can have
        parameters. \zrmref{p:genuse}

\item[]{\bf Warning -- implicitly quantified name $x$ appears only once}%
	\index{implicit quantifiers}\\
        Under the \verb/-q/ switch, an implicit quantifier has been
        inserted for the name $x$, which occurs in a predicate only
        once.  This is allowed, but may indicate a spelling mistake.

\item[]
{\bf Warning -- infix relation symbol $\omega$ is not declared as a
        binary\hfil\break\hbox{}\qquad relation}\\
{\bf Warning -- infix function symbol $\omega$ is not declared as a
        binary\hfil\break\hbox{}\qquad function}\\
{\bf Warning -- postfix function symbol $\omega$ is not declared as
        a\hfil\break\hbox{}\qquad function}\\
{\bf Warning -- prefix relation symbol $\omega$ is not declared as a set}\\
        The symbol $\omega$ has been declared with a type that is
        not appropriate to its syntactic class.  This is allowed, but
        will cause an error if the symbol is ever used.

\item[]{\bf Warning -- $X$ already declared as a basic type}\\
        The identifier $X$ appears to the left of $::=$, but has
        been declared as a basic type earlier in the specification.
        \Fuzz\ allows this so that mutually recursive free type
        definitions can be written.

\end{trivlist}
\bigskip
\noindent
\def\errtype{{\tt *errtype*}}
If there are errors in the specification, the fictitious type
\verb/*errtype*/\index{\errtype} may appear in the types reported
for variables, or in error messages. This indicates that the type
checker could not ascribe 
a type to some expression in the specification, and replaced its type
with a special marker. The type \verb/*errtype*/ never appears in the
output produced from a correct specification; it is simply a device
which allows more checking to be done on a specification after errors
have been detected.

Types appearing in error messages may also contain `place-markers',
indicated by \verb/?/~.\index{|?| as type} These place-markers are
generated when generic constants are written without explicit
parameters, and the type checker works out from the context what the
parameters must be.  If \verb/?/ appears in an error message, it
means that the type checker has not completely worked out the type
of a phrase, but the type it has found so far has the wrong shape to
match the context.  For example, the predicate
\begin{zed}
        \emptyset = 3
\end{zed}
results in the error message
\begin{verbatim}
Types do not agree in equation
> Predicate: \emptyset = 3
> LHS type:  P ?
> RHS type:  NN
\end{verbatim}
This indicates that $\emptyset$ always has a set type, but
whatever type is given to the elements of the set, this cannot be
the same as the type $\nat$ ascribed to $3$.

\chapter{Syntax Summary}\label{syntax}

This chapter contains a syntax summary of the input language
recognized by the \fuzz\ type checker, showing the \LaTeX\ commands
for the constructs of Z rather than the way they look when printed.
Except for the use of \LaTeX\ commands in place of the printed
symbols, this syntax is identical with the one in the \ZRM.
Only the formal parts of the specification are shown in the summary,
but of course one {\sf Paragraph} should be separated from the next
by plenty of explanatory text.

The following conventions about repeated and optional phrases are
used: {\sf S{\tt ,} \dots{\tt ,} S} stands for a list of one or more
instances of the class {\sf S} separated by commas, and {\sf S{\tt
;} \dots{\tt ;} S} stands for one or more instances of {\sf S}
separated by semicolons.  The notation {\sf S \dots\ S} stands for one
or more adjacent instances of {\sf S} with no separators.  Phrases
enclosed in slanted square brackets \lopt\ \dots\ \ropt\ are optional.

Certain collections of symbols have a range of binding powers: they
are the logical connectives, used in predicates and schema
expressions, the special-purpose schema operators, and infix function
symbols, used in expressions. The relative binding powers of the
logical connectives are indicated by listing them in decreasing order
of binding power; the binding powers of infix function symbols are
given in Section~\ref{symtabs}. Each production for which a binding
power is relevant has been marked with an upper-case letter at the
right margin; `L' marks a symbol which associates to the left -- so $A
\land B \land C$ means $(A \land B) \land C$ -- and `R' marks a symbol
which associates to the right. Unary symbols are marked with `U'.
\par\vfill\break % PAGE
\begin{makeatletter}% The big hack in search of Hades ...
\catcode`\`=\active \def`{\verb'}
\zedindent=0pt
\@zed\@znoskip
\halign to\linewidth\bgroup
        \strut\sf#\hfil&\hfil${}#{}$\hfil&\sf#\hfil
        \tabskip=0pt plus1fil&\hbox to0pt{\hss#\unskip}\tabskip=0pt\cr
Specification & ::= & Paragraph \dots\ Paragraph
\also
Paragraph
        & ::= & Unboxed-Para \\
        &  |  & Axiomatic-Box \\
        &  |  & Schema-Box \\
        &  |  & Generic-Box 
\also
Unboxed-Para
        & ::= & `\begin{zed}' \\
        &     & \qquad Item Sep \dots\ Sep Item \\
        &     & `\end{zed}'
\also
Item
        & ::= & `['Ident`,' \dots`,' Ident`]' \\
        &  |  & Schema-Name\lopt Gen-Formals\ropt\ `\defs' Schema-Exp \\
        &  |  & Def-Lhs `==' Expression \\
        &  |  & Ident `::=' Branch `|' \ldots `|' Branch \\
        &  |  & Predicate
\also
Axiomatic-Box 
        & ::= & `\begin{axdef}' \\
        &     & \qquad Decl-Part \\
        &     & \llap{\lopt\quad}`\where' \\
        &     & \qquad Axiom-Part\quad\ropt \\
        &     & `\end{axdef}'
\also
Schema-Box 
        & ::= & `\begin{schema}{'Schema-Name`}'\lopt Gen-Formals\ropt \\
        &     & \qquad Decl-Part \\
        &     & \llap{\lopt\quad}`\where' \\
        &     & \qquad Axiom-Part\quad\ropt \\
        &     & `\end{schema}'
\also
Generic-Box 
        & ::= & `\begin{gendef}'\lopt Gen-Formals\ropt \\
        &     & \qquad Decl-Part \\
        &     & \llap{\lopt\quad}`\where' \\
        &     & \qquad Axiom-Part\quad\ropt \\
        &     & `\end{gendef}'
\also
Decl-Part & ::= & Basic-Decl Sep \dots\ Sep Basic-Decl
\also
Axiom-Part & ::= & Predicate Sep \dots\ Sep Predicate
\also
Sep & ::= & `;' $|$ `\\' $|$ `\also'
\also
Def-Lhs
        & ::= & Var-Name\lopt Gen-Formals\ropt \\
        &  |  & Pre-Gen Decoration Ident \\
        &  |  & Ident In-Gen Decoration Ident
\also
Branch
        & ::= & Ident \\
        &  |  & Var-Name `\ldata' Expression `\rdata'
\also
Schema-Exp
        & ::= & `\forall' Schema-Text `@' Schema-Exp \\
        &  |  & `\exists' Schema-Text `@' Schema-Exp \\
        &  |  & `\exists_1' Schema-Text `@' Schema-Exp \\
        &  |  & Schema-Exp-1
\also
Schema-Exp-1
        & ::= & `[' Schema-Text `]' \\
        &  |  & Schema-Ref \\
        &  |  & `\lnot' Schema-Exp-1                            & U \\
        &  |  & `\pre' Schema-Exp-1                             & U \\
        &  |  & Schema-Exp-1 `\land' Schema-Exp-1               & L \\
        &  |  & Schema-Exp-1 `\lor' Schema-Exp-1                & L \\
        &  |  & Schema-Exp-1 `\implies' Schema-Exp-1            & R \\
        &  |  & Schema-Exp-1 `\iff' Schema-Exp-1                & L \\
        &  |  & Schema-Exp-1 `\project' Schema-Exp-1            & L \\
        &  |  & Schema-Exp-1 `\hide' \\
        &     & \qquad `('Decl-Name`,' \dots`,' Decl-Name`)' \quad & L \\
        &  |  & Schema-Exp-1 `\semi' Schema-Exp-1               & L \\
        &  |  & Schema-Exp-1 `\pipe' Schema-Exp-1               & L \\
        &  |  & `(' Schema-Exp `)'
\also
Schema-Text & ::= & Declaration \lopt `|' Predicate\ropt
\also
Schema-Ref & ::= & Schema-Name Decoration 
                        \lopt Gen-Actuals\ropt \lopt Renaming\ropt
\also
Renaming & ::= & `['Decl-Name`/'Decl-Name`,' 
                        \dots`,' Decl-Name`/'Decl-Name`]'\hidewidth
\also
Declaration & ::= & Basic-Decl`;' \dots`;' Basic-Decl
\also
Basic-Decl
        & ::= & Decl-Name`,' \dots`,' Decl-Name `:' Expression \\
        &  |  & Schema-Ref
\also
Predicate
        & ::= & `\forall' Schema-Text `@' Predicate \\
        &  |  & `\exists' Schema-Text `@' Predicate \\
        &  |  & `\exists_1' Schema-Text `@' Predicate \\
        &  |  & `\LET' Let-Def`;' \dots`;' Let-Def `@' Predicate \\
        &  |  & Predicate-1
\also
Predicate-1
        & ::= & Expression Rel Expression Rel \dots\ Rel Expression \\
        &  |  & Pre-Rel Decoration Expression \\
        &  |  & Schema-Ref \\
        &  |  & `\pre' Schema-Ref \\
        &  |  & `true' \\
        &  |  & `false' \\
        &  |  & `\lnot' Predicate-1                             & U \\
        &  |  & Predicate-1 `\land' Predicate-1                 & L \\
        &  |  & Predicate-1 `\lor' Predicate-1                  & L \\
        &  |  & Predicate-1 `\implies' Predicate-1              & R \\
        &  |  & Predicate-1 `\iff' Predicate-1                  & L \\
        &  |  & ( Predicate )
\also
Rel & ::= & `=' $|$ `\in' $|$ In-Rel Decoration $|$ `\inrel{' Ident `}'
\also
Let-Def & ::= & Var-Name `==' Expression
\also
Expression-0
        & ::= & `\lambda' Schema-Text `@' Expression \\
        &  |  & `\mu' Schema-Text \lopt`@' Expression\ropt \\
        &  |  & `\LET' Let-Def`;' \dots`;' Let-Def `@' Expression \\
        &  |  & Expression
\also
Expression
        & ::= & `\IF' Predicate `\THEN' Expression `\ELSE' Expression \\
        &  |  & Expression-1
\also
Expression-1
        & ::= & Expression-1 In-Gen Decoration Expression-1     & R \\
        &  |  & Expression-2 `\cross' Expression-2 \\
        &     & \qquad `\cross' \dots\ `\cross' Expression-2 \\
        &  |  & Expression-2
\also
Expression-2
        & ::= & Expression-2 In-Fun Decoration Expression-2     & L \\
        &  |  & `\power' Expression-4 \\
        &  |  & Pre-Gen Decoration Expression-4 \\
        &  |  & `-' Decoration Expression-4 \\
        &  |  & Expression-4 `\limg' Expression-0 `\rimg' Decoration \\
        &  |  & Expression-3
\also
Expression-3
        & ::= & Expression-3 Expression-4 \\
        &  |  & Expression-4
\also
Expression-4
        & ::= & Var-Name\lopt Gen-Actuals\ropt \\
        &  |  & Number \\
        &  |  & Schema-Ref \\
        &  |  & Set-Exp \\
        &  |  & `\langle'\lopt Expression`,' 
                        \dots`,' Expression\ropt`\rangle' \\
        &  |  & `\lbag'\lopt Expression`,' 
                        \dots`,' Expression\ropt`\rbag' \\
        &  |  & `(' Expression`,' \dots`,' Expression `)' \\
        &  |  & `\theta' Schema-Name Decoration \lopt Renaming\ropt \\
        &  |  & Expression-4 `.' Var-Name \\
        &  |  & Expression-4 Post-Fun Decoration \\
        &  |  & Expression-4 `\bsup' Expression `\esup' \\
        &  |  & `(' Expression-0 `)'
\also
Set-Exp
        & ::= & `\{' \lopt Expression`,' \dots`,' Expression\ropt\ `\}' \\
        &  |  & `\{' Schema-Text \lopt`@' Expression \ropt\ `\}'
\also
Ident & ::= & Word Decoration
\also
Decl-Name & ::= & Ident $|$ Op-Name
\also
Var-Name & ::= & Ident $|$ `('Op-Name`)'
\also
Op-Name
        & ::= & `\_' In-Sym Decoration `\_' \\
        &  |  & Pre-Sym Decoration `\_' \\
        &  |  & `\_' Post-Sym Decoration \\
        &  |  & `\_' `\limg' `\_' `\rimg' Decoration \\
        &  |  & `-' Decoration
\also
In-Sym & ::= & In-Fun $|$ In-Gen $|$ In-Rel
\also
Pre-Sym & ::= & Pre-Gen $|$ Pre-Rel
\also
Post-Sym & ::= & Post-Fun
\also
Decoration & ::= & \lopt Stroke \dots\ Stroke\ropt
\also
Gen-Formals & ::= & `['Ident`,' \dots`,' Ident`]'
\also
Gen-Actuals & ::= & `['Expression`,' \dots`,' Expression`]'
\crcr\egroup\postdisplaypenalty=-10000$$%
\end{makeatletter}%
The syntax summary uses several classes of terminal symbol that
are defined as follows:
\begin{itemize}
\item a {\sf Word}\index{identifier: syntax rules} is an undecorated
        name or special symbol. It may be either a non-empty
        sequence of letters, digits and underscores (written using
        the \verb/\_/ command) that starts with a letter, a
        non-empty sequence of characters drawn from the list
        \verb/+-*.=<>/, or a \LaTeX\ command.  Some strings that
        would otherwise be {\sf Word}'s are reserved for other
        purposes, and others are taken as {\sf Schema-Name}'s or
        operator symbols.

\item a {\sf Schema-Name}\label{lexis}%
        \index{schema name: syntax rules} is either a {\sf Word} that
        has been defined as a schema, or one of the Greek
        letters \verb/\Delta/ or \verb/\Xi/ followed by a single
        space and a {\sf Word}.

\item the classes {\sf In-Fun}, {\sf Pre-Rel}, etc., stand for
        members of the class {\sf Word} that have been announced as
        infix function symbols, prefix relation symbols, etc.,
        either in the prelude or by an explicit directive.

\item a {\sf Number} is a non-empty sequence of decimal digits.

\item a {\sf Stroke} is a single decoration: one of \verb/'/,
        \verb/?/, \verb/!/, or a subscript digit entered as
        \verb/_0/, \verb/_1/, and so on.
\end{itemize}

\vfill
{\small\center Another essential Mikro{\it nella\/} design\par}

\input index
\end{document}
